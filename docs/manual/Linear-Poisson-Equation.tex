% -*- coding: utf-8 -*-
%
% Copyright © 2012-2016 Philipp Büttgenbach
%
% This file is part of ulphi, a CAE
% tool and library for computing electromagnetic fields.
%
% Permission is granted to copy, distribute and/or modify this
% document under the terms of the \gls{gnu} Free Documentation
% License, Version 1.3 or any later version published by the Free
% Software Foundation; with no Invariant Sections, no Front-Cover
% Texts, and no Back-Cover Texts.  A copy of the license is included
% in the section entitled GNU Free Documentation License.
%
% This manual is distributed in the hope that it will be useful, but
% WITHOUT ANY WARRANTY; without even the implied warranty of
% MERCHANTABILITY or FITNESS FOR A PARTICULAR PURPOSE.
%

\chapter{Linear \personname{Poisson} Equation}

\section{Electrostatics}
\label{sec:electrostatics}

The \personname{Poisson} equation reads:
\begin{align}
  \laplaceop \phi &= \frac{\rho}{\epsilon} \\
  \Leftrightarrow\quad\divop \grad \phi &= \frac{\rho}{\epsilon}
\end{align}
This equation is integrated by the theorem of \personname{Gauß}:
\begin{gather}
  \epsilon\oiinteg{S}{\grad \phi}{S} = \iiinteg{V}{\rho}{V} = Q
  \label{eq:20}
\end{gather}
The $\grad$-operator is kept and not replaced by an integral
expression.  This is alright because keeping this operator doesn't
introduce any averaging of grid edges so that the advantages of using
\gls{fit} pointed out at the begining of
chapter~\ref{cha:discretization} are preserved.
\begin{figure}
  \centering
%  \input{../images/Poisson-grid-3D.tex}
  \caption{A basic grid cell in 3D}
  \label{fig:BasicGridCellPoisson3D}
\end{figure}
Figure~\ref{fig:BasicGridCellPoisson3D} shows a basic grid cell.
Aplying equation~\eqref{eq:20} to this grid leads to:
\newcommand*{\phiepsilonSl}[2]{
  \ensuremath{\phi_{#1,#2}
  \frac{\epsilon_{#1,#2}\cdot S_{#1,#2}}{l_{#1,#2}}}}
\begin{multline}
  +\phiepsilonSl{13}{14}+\phiepsilonSl{13}{16} \\
  -\phiepsilonSl{12}{13}-\phiepsilonSl{10}{13} \\
  +\phiepsilonSl{13}{22}-\phiepsilonSl{4}{13} = Q
\end{multline}
with
\begin{align*}
  l&\quad \text{the grid edge'es length} \\
  S&\quad \text{Surface perpendicular to this edge}
\end{align*}
Doing so for all cells of a larger grid yields an equation system
which has the same structure as \eqref{eq:18}:
\begin{gather}
  \matr{G}\vect{\phi}=\vect{\rho}\circ\vect{V}=\vect{Q}
\end{gather}
with
\begin{gather}
  \label{eq:32}
  g=\epsilon\frac{S}{l}
\end{gather}

\section{Reduction to 2D}
\label{sec:reduction-2d}

\begin{itemize}
\item We asume the homogeneous \personname{Neumann} condition on surface
  $S_{13,22}$ and $S_{4,13}$.
\item The grid shall have the fixed length $l_D$ in $z$-direction.
\end{itemize}
It follows that the source term is a surface charge
density~\gls{sigma} and the coefficients forming the system matrix are
\begin{gather}
  \label{eq:22}
  g=\frac{\tilde{\Delta}}{\Delta}\gls{epsilon}\text{~.}
\end{gather}

\section{Thermostatics}
\label{sec:thermostatics}

\begin{gather}
  \label{eq:33}
  \laplaceop T = -\frac{1}{\lambda}\partdiff{\gls{Phith}}{V}
\end{gather}
\begin{itemize}
\item The source term is \gls{qth}
\item $g=\frac{\tilde{\Delta}}{\Delta}\gls{lambdath}$
\end{itemize}
  
%%% Local Variables: 
%%% mode: latex
%%% TeX-master: "manual"
%%% ispell-local-dictionary: "en_US"
%%% End: 
