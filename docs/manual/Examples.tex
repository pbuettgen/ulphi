% -*- coding: utf-8 -*-
%
% Copyright © 2012-2016 Philipp Büttgenbach
%
% This file is part of ulphi, a CAE tool and library for
% computing electromagnetic fields.
%
% Permission is granted to copy, distribute and/or modify this
% document under the terms of the \gls{gnu} Free Documentation
% License, Version 1.3 or any later version published by the Free
% Software Foundation; with no Invariant Sections, no Front-Cover
% Texts, and no Back-Cover Texts.  A copy of the license is included
% in the section entitled GNU Free Documentation License.
%
% This manual is distributed in the hope that it will be useful, but
% WITHOUT ANY WARRANTY; without even the implied warranty of
% MERCHANTABILITY or FITNESS FOR A PARTICULAR PURPOSE.
%

\chapter{Examples}
\label{cha:examples}

\section{Coaxial Line}
\label{sec:coaxial-line}

The geometry for this example is shown in figure~\ref{fig:coaxial-line}.
It is a good example for validating the library because an analytical
solution is known.  \par The inner conductor carries the current~$I$
and the outer conductor carries the current~$-I$ -- both homogeneously
distributed over the conductor's surface.  The current density in the
outer conducter is
\begin{gather}
  J_3=-J_1\cdot\frac{r_1^2}{r_3^2-r_2^2}\text{~.}
\end{gather}
Using equation~\eqref{eq:3} the magnetic field intensity computes to
\begin{gather}
  H(r)= \frac{J_1}{2}
  \begin{cases}
    r & r \le r_1\text{~,} \\
    \frac{r_1^2}{r} & r_1 < r \le r_2\text{~,} \\
    \frac{r_1^2}{r}\cdot\left(1-\frac{r^2-r_2^2}{r_3^2-r_2^2}\right)
    & r_2 < r \le r_3\text{~.}
  \end{cases}
\end{gather}
From \eqref{eq:4} the magnetic vector potential is found to be
\begin{multline}
\label{eq:12}
  A(0)-A(r) = \frac{\mu_0J_1}{2} \\ \cdot
  \begin{cases}
    r^2/2 & r \le r_1\text{~,} \\
    r_1^2\left(\frac{1}{2}+\ln\frac{r}{r_1}\right)
    & r_1 < r \le r_2\text{~,} \\
    r_1^2\left(\frac{1}{2}+\ln\frac{r_2}{r_1}
      +(1+\frac{r_2^2}{r_3^2-r_2^2})\ln\frac{r}{r_2}
      +\frac{r^2-r_2^2}{2(r_3^2-r_2^2)}\right)
    & r_2 < r \le r_3\text{~.} \\
  \end{cases}
\end{multline}
Both solutions -- the numerical and \eqref{eq:12} -- are compared in
figure~\ref{fig:VectPotentialCoaxialLine} and agree very well.
\begin{figure}
  \centering
  \sisetup{per-mode=fraction}
  \subcaptionbox{Geometry\label{fig:coaxial-line}}{
    \includeinkscape{Coaxial-Line}
  } \hfil
  \subcaptionbox{Result\label{fig:VectPotentialCoaxialLine}}{
    \begin{tikzpicture}
      \begin{axis}[
        width=.65\linewidth,
        height=0.40172209268743164\linewidth,
        xlabel=$r/\unit{\milli\metre}\quad\longrightarrow$,
        ylabel=$A\cdot\unit{\metre\per\micro\weber}\quad\longrightarrow$,
        grid=major,
        grid style={line width=.18mm},
        tick style={line width=.35mm},
        legend style={at={(0.95,0.95)},anchor=north east},
        line width=.35mm]
        \addplot[mark=diamond,only marks]
          table[x=x, y=NumA] {coaxial_line.csv};
        \addplot[smooth,mark=none,line width=.7mm,style=dotted]
          table[x=x, y=SymA] {coaxial_line.csv};
        \legend{numerical, analytical}
      \end{axis}
    \end{tikzpicture}
  }
  \caption{Coaxial line}
\end{figure}

\section{Simple iron core}
\label{sec:simple-iron-core}

In this example the coil is fed by a direct current.  After the
computation is done, the result is checked against the result gained
by the average-iron-path-method.  Also the field line plot
(figure~\ref{fig:SimpleIronCoreFieldLines}) looks reasonable.
\begin{figure}
  \centering
  \sisetup{per-mode=fraction}
  \begin{tikzpicture}
    \begin{axis}[
      title={$A\cdot\unit{\metre\per\milli\weber}$},
      width=.95\linewidth,
      axis equal={true},
      xmin={0},
      xmax=50,
      ymin=-5,
      ymax=60,
      % xtick={-10,0,...,60},
      % ytick={-10,0,...,60},
      xlabel=$x/\unit{\milli\metre}\quad\longrightarrow$,
      ylabel=$y/\unit{\milli\metre}\quad\longrightarrow$,
      grid=major,
      grid style={line width=.18mm},
      tick style={line width=.35mm},
      line width=.35mm]
      %% iron core shape
      \addplot[gray,fill=gray!20,opacity=.33,line width=.7mm] 
        coordinates {
          (15,0) (40,0) (40,54) (0, 54) (0, 29) (15,29)} -- cycle;
      %% coils
      \addplot[copper,fill=copper,opacity=.33,line width=.7mm]
        coordinates {(5,0) (10,0) (10,24) (5,24)} -- cycle;
      \addplot[copper,fill=copper,opacity=.33,line width=.7mm]
        coordinates { (45,0) (50,0) (50,24) (45,24)} -- cycle;
      % \addplot[contour lua, line width=.7mm]
      %   table[x=x, y=y, z=potential] {simple_iron_core.csv};
    \end{axis}
  \end{tikzpicture}
  \caption{Field lines inside the simple iron core}
  \label{fig:SimpleIronCoreFieldLines}
\end{figure}


\section{E-I-Transformer core}
\label{sec:inductor}

The device and the model setup for this example are shown in
figure~\ref{fig:ei-transformer-device} and
\ref{fig:ei-transformer-model}.
\begin{figure}
  \hfil
  \subcaptionbox{physical device\label{fig:ei-transformer-device}}{
    \includeinkscape{Mantel-Transformator}
  } \hfil
  \subcaptionbox{Model\label{fig:ei-transformer-model}}{
    \includeinkscape{Mantel-Transformator-Modell}
  } \hfil
  \\[\bigskipamount]
  \begin{minipage}{\textwidth}
  \centering
    \subcaptionbox{
      average core flux density\label{fig:ei-transformer-result-time-domain}}{
      \begin{tikzpicture}
        \begin{axis}[
          width=.6\linewidth,
          height=0.3708203932499369\linewidth,
          xlabel=$t/\unit{\milli\second}\quad\longrightarrow$,
          ylabel=$B/\unit{\tesla}\quad\longrightarrow$,
          grid=major, grid style={line width=.18mm},
          tick style={line width=.35mm},
          legend style={at={(0.95,0.95)},anchor=north east},
          line width=.35mm]
          \addplot[line width=.7mm]
            table[x=t, y=y1]{E_I_transformer_core_time.csv}; 
          \addplot[smooth, line width=.7mm, style=dashed]
            table[x=t, y=y2]{E_I_transformer_core_time.csv}; 
          \legend{waveform, fundamental}
        \end{axis}
      \end{tikzpicture}
    } \\ \bigskip
    \subcaptionbox{spectrum\label{fig:ei-transformer-result-harmonics}}{
      \begin{tikzpicture}
        \begin{axis}[ 
          width=.6\linewidth,
          height=0.3708203932499369\linewidth,
          xlabel=harmonic order$\quad\longrightarrow$,
          ylabel=$B/\unit{\tesla}\quad\longrightarrow$,
          grid=major, grid style={line width=.18mm},
          tick style={line width=.35mm}, line width=.35mm]
          \addplot[ycomb, line width=.7mm]
            table{E_I_transformer_core_freq.csv};
        \end{axis}
      \end{tikzpicture}
    }
  \end{minipage}
  \caption{E-I-transformer core}
  \label{fig:transformer}
\end{figure}
In this example the coil is fed by an alternating current.  As the
results show (figure~\ref{fig:ei-transformer-result-time-domain} and
\ref{fig:ei-transformer-result-harmonics}) the device is already
operated at the level of saturation.

\section{Wound Iron Core}
\label{sec:wound-iron-core}

This is the example which motivated the development of this library.
The iron core (figure~\ref{fig:wound-transformer-core}) has straight
and rounded sections.  This can be modeled using the different types
of grids.  \par In all examples before a current source
was used.  This is very uncommon in practice: Usually you are using a
voltage source.  Also you cannot specify a voltage source in ulphi,
you can get the same effect by using a inhomogeneous
\personname{Dirichlet} boundary condition to specify the total flux
passing throw the iron core.  After solving the task the magnetic
voltage is evaluated.
\begin{figure}
  \hfil
  \subcaptionbox{physical device}{
    \includeinkscape{Wound-Transformer-Core}
  }
  \hfil
  \subcaptionbox{model}{
    \includeinkscape{wound-transformer-core-model}
  }
  \hfil
  \caption{Wound iron core}
\end{figure}
\begin{figure}
\centering
  \subcaptionbox{magnetic voltage\label{fig:wound-iron-core-result-time-domain}}{
    \begin{tikzpicture}
      \begin{axis}[
        width=.85\linewidth,
        height=0.5253288904374106\linewidth,
        xlabel=$t/\unit{\milli\second}\quad\longrightarrow$,
        ylabel=$\gls{Um}/\unit{\kilo\ampere}\quad\longrightarrow$,
        grid=major, grid style={line width=.18mm},
        tick style={line width=.35mm},
        legend style={at={(0.95,0.95)},anchor=north east},
        line width=.35mm]
        \addplot[line width=.7mm]
          table[x=t, y=y1]{wound_iron_core_time.csv}; 
        \addplot[smooth, line width=.7mm, style=dashed]
          table[x=t, y=y2]{wound_iron_core_time.csv};
        \legend{waveform, fundamental}
      \end{axis}
    \end{tikzpicture}
  }
  \\[\bigskipamount]
  \subcaptionbox{spectrum\label{fig:wound-iron-core-result-harmonics}}{
    \begin{tikzpicture}
      \begin{axis}[
        width=.85\linewidth,
        height=0.5253288904374106\linewidth,
        xlabel=harmonic order$\quad\longrightarrow$,
        ylabel=$\gls{Um}/\unit{\kilo\ampere}\quad\longrightarrow$,
        grid=major, grid style={line width=.18mm},
        tick style={line width=.35mm},
        line width=.35mm]
        \addplot[ycomb, line width=.7mm]
          table{wound_iron_core_freq.csv};
      \end{axis}
    \end{tikzpicture}
  }
  \caption{Wound iron core calculation results}
  \label{fig:wound-transformer-core}
\end{figure}


%%% Local Variables: 
%%% mode: latex
%%% TeX-master: "manual"
%%% ispell-local-dictionary: "en_US"
%%% End: 
