% -*- coding: utf-8 -*-
%
% Copyright © 2012-2016 Philipp Büttgenbach
%
% This file is part of ulphi, a CAE tool and library for
% computing electromagnetic fields.
%
% Permission is granted to copy, distribute and/or modify this
% document under the terms of the \gls{gnu} Free Documentation
% License, Version 1.3 or any later version published by the Free
% Software Foundation; with no Invariant Sections, no Front-Cover
% Texts, and no Back-Cover Texts.  A copy of the license is included
% in the section entitled GNU Free Documentation License.
%
% This manual is distributed in the hope that it will be useful, but
% WITHOUT ANY WARRANTY; without even the implied warranty of
% MERCHANTABILITY or FITNESS FOR A PARTICULAR PURPOSE.
%

\chapter{Introduction}
\label{cha:introduction}

There are already some tools available for doing electromagnetic
computations.  So, why start a new one?  The point is that most tools
have only limited support for modeling anisotropic materials.
Unfortunately such materials commonly appear in electromagnetic
applications: All iron cores are essentially anisotropic materials.
Nevertheless in many cases (depending on flux direction) an isotropic
material is sufficiant, but there are other cases (like wound iron
cores) where strong support for anisotropic materials is required.
So, the goal of this project is to establish a tool which takes
special care of these anisotropic materials and makes their modeling
easy. \par
The average iron path method (figure~\ref{fig:descret-meth-1}) dates
back to the early ages of electrical engineering.  It is based on
equation~\eqref{eq:3} in its integral form.
\begin{description}
\item[Advantages] \hfill
  \begin{itemize}
  \item The average iron path follows expected flux lines.
  \item It is easy to handle straight and rounded sections within one
    calculation.
  \end{itemize}
\item[Disadvantages] \hfill
  \begin{itemize}
  \item The method presumes a homogeneous field distribution and a
    homogeneous material.
  \item The local field distribution stays unknown.
  \end{itemize}
\end{description}
These disadvantages can be overcome by using numerical methods like
the \gls{fem} (figure~\ref{fig:descret-meth-2}).
\begin{description}
\item[Advantages] \hfill
  \begin{itemize}
  \item Local field distribution is computed.
  \item Magnetic reluctivity may be inhomogeneous.
  \end{itemize}
\item[Disadvantages] \hfill
  \begin{itemize}
  \item An anisotropic reluctivity must be interpolated to match the
    element's coordinate system.
  \item Rough results in the rounded section.
  \end{itemize}
\end{description}
The goal of this work is to combine the advantages of both
aforementioned methods and to avoid their disadvantages.  This is done
by using an integral method -- the \gls{fit} -- on a regular grid which is
well aligned with the expected flux lines and the material's axis of
anisotropy (figure~\ref{fig:descret-meth-3}).
\begin{figure}
  \centering
  \subcaptionbox{average iron path\label{fig:descret-meth-1}}{
    \includeinkscape{Compute-with-av-path}
  }\\ \bigskip
  \subcaptionbox{typical FEM-mesh\label{fig:descret-meth-2}}{
    \includeinkscape{Compute-with-fem}
  }\\ \bigskip
  \subcaptionbox{FIT grid\label{fig:descret-meth-3}}{
    \includeinkscape{Compute-with-fit}
  }
  \caption{Discretization methods}
  \label{fig:descret-meth}
\end{figure}

%%% Local Variables: 
%%% mode: latex
%%% TeX-master: "manual"
%%% ispell-local-dictionary: "en_US"
%%% End: 
