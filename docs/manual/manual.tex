% -*- coding: utf-8 -*-
%
% Copyright © 2012-2024 Philipp Büttgenbach
%
% This file is part of ulphi, a CAE tool and library for computing
% electromagnetic fields.
%
% Permission is granted to copy, distribute and/or modify this
% document under the terms of the \gls{gnu} Free Documentation
% License, Version 1.3 or any later version published by the Free
% Software Foundation; with no Invariant Sections, no Front-Cover
% Texts, and no Back-Cover Texts.  A copy of the license is included
% in the section entitled GNU Free Documentation License.

\NeedsTeXFormat{LaTeX2e}[2022-11-01]

\documentclass[paper=b5,fontsize=11pt,twoside,headsepline=yes,DIV=11]{scrreprt}

\usepackage{subcaption}
\usepackage[colorlinks]{hyperref}
\usepackage[acronyms,xindy]{glossaries}
\usepackage[algoruled]{algorithm2e}
\usepackage{listings}
\usepackage{booktabs}

% -*- coding: utf-8 -*-
%
% Copyright © 2012-2015 Philipp Büttgenbach
%
% This file is part of ulphi, a CAE tool and library for computing
% electromagnetic fields.
%
% Permission is granted to copy, distribute and/or modify this
% document under the terms of the \gls{gnu} Free Documentation
% License, Version 1.3 or any later version published by the Free
% Software Foundation; with no Invariant Sections, no Front-Cover
% Texts, and no Back-Cover Texts.  A copy of the license is included
% in the section entitled GNU Free Documentation License.
%

\PassOptionsToPackage{math-style=ISO,partial=upright}{unicode-math}

\usepackage[ngerman,main=english]{babel}
\usepackage{amsmath,amsthm,amssymb}
\usepackage{siunitx}
\usepackage{graphicx,svg}
\usepackage{pgfplots}
\usepackage{csquotes}
\usepackage[style=authoryear]{biblatex}
\usepackage{sourcesanspro}
\usepackage[stixtwo]{fontsetup}

\newcommand*{\personname}[1]{\textsc{#1}}
\newcommand*{\partdiff}[2]{\ensuremath{\frac{\partial#1}{\partial#2}}}
\newcommand*{\aninteg}[4]{\ensuremath{#1_{#2}#3\;\symrm{d}#4}}
\newcommand*{\integ}[3]{\aninteg{\int}{#1}{#2}{#3}}
\newcommand*{\integg}[4]{\ensuremath{\int_{#1}^{#2}#3\;\symrm{d}#4}}
\newcommand*{\iinteg}[3]{\aninteg{\iint}{#1}{#2}{#3}}
\newcommand*{\iiinteg}[3]{\aninteg{\iiint}{#1}{#2}{#3}}
\newcommand*{\ointeg}[3]{\ensuremath{\oint_{#1}\,#2\;\symrm{d}#3}}
\newcommand*{\oiinteg}[3]{\aninteg{\oiint}{#1}{#2}{#3}}
\newcommand*{\vect}[1]{\ensuremath{\symbf{#1}}}
\newcommand*{\matr}[1]{\vect{#1}}
\newcommand*{\jacmatr}{\ensuremath{\matr{G}_{\symrm{J}}}}
\newcommand*{\xmax}[1]{\ensuremath{#1_{\text{max}}}}
\newcommand*{\diag}{\ensuremath{\operatorname{diag}}}
\newcommand*{\vectnorm}[1]{\ensuremath{\lvert#1\rvert}}
\newcommand*{\laplaceop}{\ensuremath{\operatorname{\increment}}}
\newcommand*{\grad}{\ensuremath{\operatorname{grad}}}
\newcommand*{\divop}{\ensuremath{\operatorname{div}}}

\addbibresource{../include/literature.bib}

\graphicspath{{../images}}
\svgpath{{../images/}}

\pgfplotsset{
  compat=1.8,
  grid=major,
  grid style={line width=.18mm},
  tick style={line width=.35mm},
  table/search path={../datasets},
}

\definecolor{copper}{cmyk}{0,.32,.72,.28}

\input{../include/config.tex}

%%% Local Variables: 
%%% mode: LaTeX
%%% ispell-local-dictionary: "en_US"
%%% End:


\setkomafont{descriptionlabel}{\sourcesansprolight\bfseries}

\lstset{texcl=true,frame=tb,aboveskip=0mm,belowskip=0mm,boxpos=t}

\hypersetup{
  pdftitle={\ProjectName{} User Manual},
  pdfauthor={Philipp Büttgenbach},
  pdfkeywords={Finite integrals, anisotropic magnetic, transformer core},
  pdflang=en}

\pagestyle{headings}

\newcommand*{\Rm}{\gls{Rm}}
\newcommand*{\kFe}{\gls{kFe}}
\newcommand*{\dsheet}{\ensuremath{d_{\text{sheet}}}}
\newcommand*{\xAir}[1]{\ensuremath{#1(\text{Air}})}
\newcommand*{\xFe}[1]{\ensuremath{#1(\text{Fe}})}
\newcommand*{\RmAir}{\xAir{\gls{Rm}}}
\newcommand*{\RmFe}{\xFe{\gls{Rm}}}
\newcommand*{\nuAir}{\xAir{\gls{nu}}}
\newcommand*{\nuFe}{\xFe{\gls{nu}}}
\newcommand*{\SAir}{\xAir{S}}
\newcommand*{\SFe}{\xFe{S}}
\newcommand*{\nueffpar}{\ensuremath{\nu_{\mathrm{eff}\,\parallel}}}
\newcommand*{\nueffperp}{\ensuremath{\nu_{\mathrm{eff}\,\perp}}}
\newcommand*{\Um}{\gls{Um}}
\newcommand*{\rin}{\ensuremath{r_{\text{in}}}}
\newcommand*{\rout}{\ensuremath{r_{\text{out}}}}
\newcommand*{\pythonlogo}{
    \includeinkscape[width=2.4em]{python-logo-only}
}

\makeglossaries
\renewcommand{\entryname}{Notation}
\renewcommand{\descriptionname}{Function Name}
\renewcommand{\pagelistname}{Number of Formula}
\loadglsentries{Notation}

\newacronym{cad}{CAD}{computer aided design}
\newacronym{fdm}{FDM}{finite difference method}
\newacronym{fem}{FEM}{finite element method}
\newacronym{fd}{fd}{finite differences}
\newacronym{fit}{FIT}{finite integration technique}
\newacronym{fi}{FI}{finite integrals}
\newacronym{gnu}{GNU}{GNU’s Not Unix}
\newacronym{llc}{LLC}{lower left corner}
\newacronym{urc}{URC}{upper right corner}

\begin{document}
\title{ \includeinkscape{Ulphi-logo} }
\subtitle{User Manual}
\author{Philipp Büttgenbach}
\lowertitleback{
  Copyright © 2012-2024 Philipp Büttgenbach

  Permission is granted to copy, distribute and/or modify this
  document under the terms of the \gls{gnu} Free Documentation
  License, Version 1.3 or any later version published by the Free
  Software Foundation; with no Invariant Sections, no Front-Cover
  Texts, and no Back-Cover Texts.  A copy of the license is included
  in section~\ref{cha:gnu-free-docum}.

  This manual is distributed in the hope that it will be useful, but
  \emph{without any warranty}; without even the implied warranty of
  \emph{merchantability} or \emph{fitness for a particular purpose}.
}
\maketitle
\tableofcontents
\listoffigures
\printglossary[title={Index of Notations}]
\printacronyms

% -*- coding: utf-8 -*-
%
% Copyright © 2012-2016 Philipp Büttgenbach
%
% This file is part of ulphi, a CAE tool and library for
% computing electromagnetic fields.
%
% Permission is granted to copy, distribute and/or modify this
% document under the terms of the \gls{gnu} Free Documentation
% License, Version 1.3 or any later version published by the Free
% Software Foundation; with no Invariant Sections, no Front-Cover
% Texts, and no Back-Cover Texts.  A copy of the license is included
% in the section entitled GNU Free Documentation License.
%
% This manual is distributed in the hope that it will be useful, but
% WITHOUT ANY WARRANTY; without even the implied warranty of
% MERCHANTABILITY or FITNESS FOR A PARTICULAR PURPOSE.
%

\chapter{Introduction}
\label{cha:introduction}

There are already some tools available for doing electromagnetic
computations.  So, why start a new one?  The point is that most tools
have only limited support for modeling anisotropic materials.
Unfortunately such materials commonly appear in electromagnetic
applications: All iron cores are essentially anisotropic materials.
Nevertheless in many cases (depending on flux direction) an isotropic
material is sufficiant, but there are other cases (like wound iron
cores) where strong support for anisotropic materials is required.
So, the goal of this project is to establish a tool which takes
special care of these anisotropic materials and makes their modeling
easy. \par
The average iron path method (figure~\ref{fig:descret-meth-1}) dates
back to the early ages of electrical engineering.  It is based on
equation~\eqref{eq:3} in its integral form.
\begin{description}
\item[Advantages] \hfill
  \begin{itemize}
  \item The average iron path follows expected flux lines.
  \item It is easy to handle straight and rounded sections within one
    calculation.
  \end{itemize}
\item[Disadvantages] \hfill
  \begin{itemize}
  \item The method presumes a homogeneous field distribution and a
    homogeneous material.
  \item The local field distribution stays unknown.
  \end{itemize}
\end{description}
These disadvantages can be overcome by using numerical methods like
the \gls{fem} (figure~\ref{fig:descret-meth-2}).
\begin{description}
\item[Advantages] \hfill
  \begin{itemize}
  \item Local field distribution is computed.
  \item Magnetic reluctivity may be inhomogeneous.
  \end{itemize}
\item[Disadvantages] \hfill
  \begin{itemize}
  \item An anisotropic reluctivity must be interpolated to match the
    element's coordinate system.
  \item Rough results in the rounded section.
  \end{itemize}
\end{description}
The goal of this work is to combine the advantages of both
aforementioned methods and to avoid their disadvantages.  This is done
by using an integral method -- the \gls{fit} -- on a regular grid which is
well aligned with the expected flux lines and the material's axis of
anisotropy (figure~\ref{fig:descret-meth-3}).
\begin{figure}
  \centering
  \subcaptionbox{average iron path\label{fig:descret-meth-1}}{
    \includeinkscape{Compute-with-av-path}
  }\\ \bigskip
  \subcaptionbox{typical FEM-mesh\label{fig:descret-meth-2}}{
    \includeinkscape{Compute-with-fem}
  }\\ \bigskip
  \subcaptionbox{FIT grid\label{fig:descret-meth-3}}{
    \includeinkscape{Compute-with-fit}
  }
  \caption{Discretization methods}
  \label{fig:descret-meth}
\end{figure}

%%% Local Variables: 
%%% mode: latex
%%% TeX-master: "manual"
%%% ispell-local-dictionary: "en_US"
%%% End: 

% -*- coding: utf-8 -*-
%
% Copyright © 2012-2016 Philipp Büttgenbach
%
% This file is part of ulphi, a CAE
% tool and library for computing electromagnetic fields.
%
% Permission is granted to copy, distribute and/or modify this
% document under the terms of the \gls{gnu} Free Documentation
% License, Version 1.3 or any later version published by the Free
% Software Foundation; with no Invariant Sections, no Front-Cover
% Texts, and no Back-Cover Texts.  A copy of the license is included
% in the section entitled GNU Free Documentation License.
%
% This manual is distributed in the hope that it will be useful, but
% WITHOUT ANY WARRANTY; without even the implied warranty of
% MERCHANTABILITY or FITNESS FOR A PARTICULAR PURPOSE.
%

\chapter{General Electromagnetic Theory}
\label{cha:gener-electr-theory}

Electromagnetic fields are generally described by
\personname{Maxwell}'s equations.  Using \personname{Gauss}' and
\personname{Stockes}' theorem, these equations can be converted from
their differential form into an integral form:
\begin{align}
  \label{eq:1}
  \nabla\cdot\vect{\gls{B}}&=0 &\Leftrightarrow&&
  \oiinteg{\gls{S}}{\vect{B}}{\vect{S}}&=0 \\
  %% 
  \nabla\cdot\vect{D}&=\varrho &\Leftrightarrow&&
  \oiinteg{\gls{S}}{\vect{\gls{D}}}{\vect{S}}&=\gls{Q} \\
  %% 
  \label{eq:7}
  \nabla\vectimes\vect{\gls{E}}&=-\partdiff{\vect{B}}{t} &\Leftrightarrow&&
  \ointeg{l}{\vect{\gls{E}}}{\vect{l}}&=-\partdiff{\gls{Phi}}{t} \\
  %%
  \label{eq:2}
  \nabla\vectimes\vect{\gls{H}}&=\vect{\gls{J}}+\partdiff{\vect{D}}{t}
  &\Leftrightarrow&&
  \ointeg{l}{\vect{H}}{\vect{l}}&= \gls{Theta} + \partdiff{\gls{Psi}}{t}
\end{align}
In low frequency applications, that means a frequency below about one
kilohertz, displacement currents can be generally ignored.  There\-by
equation~\eqref{eq:2} simplyfies to
\begin{gather}
  \label{eq:3}
  \nabla\vectimes\vect{H}=\vect{J}\text{~.}
\end{gather}
Next a magnetic vector potential is introduced:
\begin{gather}
  \label{eq:4}
  \nabla\vectimes\gls{A}=\vect{B}\quad\Leftrightarrow\quad
  \ointeg{l}{\vect{A}}{\vect{l}}=\Phi\text{~.}
\end{gather}
By this definition equation~\eqref{eq:1} is fulfilled automatically.
This potential is ambiguous because
\begin{gather}
  \label{eq:8}
  \nabla\vectimes\left(\vect{A}+\nabla F\right)
  =\nabla\vectimes\vect{A}+\nabla\vectimes\left(\nabla F\right)
  =\nabla\vectimes\vect{A}\text{~,}
\end{gather}
where $F$ is some scalar field.
To make this potential unique
\begin{itemize}
\item a gauge like $\nabla\cdot\vect{A}=0$ can be enforced
\item the potential may be preset at least at one point inside the
  computation domain or its boundary.  This is the case when a
  \personname{Dirichlet} boundary condition
  (section~\ref{sec:dirichlet}) is defined.
\end{itemize}
Using this potential in equation~\eqref{eq:3} yields
\begin{gather}
  \label{eq:5}
  \nabla\vectimes\left(\nu\nabla\vectimes\vect{A}\right)=\vect{J}\text{~.}
\end{gather}
\par Eddy currents are not regarded in the following in order to keep
things simple.  For many devices this is a valid procedure:
\begin{itemize}
\item Iron cores of transformers (and generally of electrical
  machines) are subdevided into single sheets to suppress eddy currents.
\item Windings are made of thin wire in order to suppress eddy currents.
\end{itemize}

%%% Local Variables: 
%%% mode: latex
%%% TeX-master: "manual"
%%% ispell-local-dictionary: "en"
%%% End: 

% -*- coding: utf-8 -*-
%
% Copyright © 2012-2016 Philipp Büttgenbach
%
% This file is part of ulphi, a CAE tool and library for
% computing electromagnetic fields.
%
% Permission is granted to copy, distribute and/or modify this
% document under the terms of the \gls{gnu} Free Documentation
% License, Version 1.3 or any later version published by the Free
% Software Foundation; with no Invariant Sections, no Front-Cover
% Texts, and no Back-Cover Texts.  A copy of the license is included
% in the section entitled GNU Free Documentation License.
%
% This manual is distributed in the hope that it will be useful, but
% WITHOUT ANY WARRANTY; without even the implied warranty of
% MERCHANTABILITY or FITNESS FOR A PARTICULAR PURPOSE.
%

\chapter{Discretization -- The  \glsentrytext{fit}}
\label{cha:discretization}

Unfortunately, the equations from the previous section can only be
solved in a limited number of cases in a closed form.  In the general
case they are discretized, that is the computation domain is
subdivided by a grid or mesh, and solved numerically. Here we are
going to use the \gls{fit}. \par
Originally the \gls{fit} was developed to compute especially high
frequency electromagnetic fields as they accure in radio frequency
particle accelerators \parencite{Weiland:1977}.  In that case it uses
the full set of equations~\eqref{eq:1} throw \eqref{eq:2}.
Nevertheless it can be seen as a much more general method and adapted
to other field equations as it is shortly pointed out by \cite[section
3.4.2]{vanRienen:2010}. \par
Compared to the \gls{fdm} it has some important advantages:
\begin{itemize}
\item The resulting system matrix is always symmetric, also for non
  equidistant grids.  This is an important property on which many
  sparse system solvers rely.
\item Coupling grids of different type (cartesian and polar) is very
  easy and natural to implement and there is a smooth transition in
  the coefficients.
\end{itemize}
\par
The finite integral method starts with the equations from section 2 in
their integral form. The computation domain is subdivided by a grid
and the grid’s edges become the integral’s integration path
(figure~\ref{fig:FIT-grid-3D}).
\begin{figure}
  \centering
  \subcaptionbox{
    Three dimensional \gls{fit} grid for equation~\eqref{eq:4}\label{fig:FIT-grid-3D}} {
    \includeinkscape{FIT-grid-3D}
  }
  \\ \bigskip
  \subcaptionbox{
    Two dimensional projection to the $x$-$y$-plane\label{fig:FIT-grid-2D}} {
    \includeinkscape{FIT-grid-2D}
  }
  \caption{Elementary \gls{fit} grid in magnetostatic applications}
\end{figure}
In the case where a computation in two dimensions is sufficient, this
grid is projected to a plane (figure~\ref{fig:FIT-grid-2D}).
All entities in this grid are referred to by their row and column index:
\begin{description}
\item[$m$] row index $m\in\left[0\dots\xmax{m}\right]$
\item[$n$] column index $n\in\left[0\dots\xmax{n}\right]$
\end{description}
Using equation~\eqref{eq:4} in its integral form the flux density
inside such a grid computes as
\begin{gather}
  \label{eq:9}
  \left[-A_z(m,n)+A_z(m+1,n)\right]\cdot\gls{lD}=\Phi_y(m,n)\text{~,}
  \intertext{and with}
  \Phi_y(m,n)=B_x(m,n)\cdot\gls{lD}\cdot\Delta_y(m)
\end{gather}
it follows
\begin{subequations}
  \begin{align}
    B_x(m,n)&=\frac{1}{\Delta_y(m)}\left[A_z(m+1,n)-A_z(m,n)\right]
    \intertext{Analogously it is}
    \label{eq:10}
    B_y(m,n)&=\frac{1}{\Delta_x(n)}\left[A_z(m,n)-A_z(m,n+1)\right]\text{~.}
  \end{align}
\end{subequations}
\par The outer rotation in equation~\eqref{eq:5} forms a
secondary or inner grid.  This grid has the following properties
\footnote{All the secondary grid's properties are marked by $\tilde{{}}$.}:
\begin{align}
  \tilde{\Delta}_{x}(n)
  &=\frac{1}{2}\cdot\left[x(n+1)-x(n-1)\right] \\
  \tilde{\Delta}_{y}(m)
  &=\frac{1}{2}\cdot\left[y(m+1)-y(m-1)\right]
  \intertext{A grid cell's area is given by}
  \label{eq:13}
  \tilde{S}(m,n)
  &=\tilde{\Delta}_x(n)\cdot\tilde{\Delta}_y(m)
\end{align}
Polar coordinates:
\begin{align}
  \label{eq:14}
  \tilde{\Delta}_{r}(n)&=\frac{1}{2}\left[r(n+1)-r(n-1)\right] \\
  \tilde{\Delta}_{\phi}(m)&=\frac{1}{2}\left[\phi(m+1)-\phi(m-1)\right]\\
  \tilde{r}(n) &= \frac{1}{2}\left[r(n+1)+r(n)\right] \\
  \tilde{S}(m,n) &= \frac{\tilde{\Delta}_{\phi}(m)}{2}
  \left[\tilde{r}^{2}(n+1)-\tilde{r}^{2}(n)\right]
\end{align}
Writing out the equations for the grid in figure~\ref{fig:FIT-grid-2D}
yields:
\begin{align*}
  B_x(\tilde{0})&=\frac{1}{\Delta_y(1)}\left[A_z(4)-A_z(1)\right] \\
  B_x(\tilde{3})&=\frac{1}{\Delta_y(4)}\left[A_z(7)-A_z(4)\right] \\
  B_y(\tilde{0})&=\frac{1}{\Delta_x(3)}\left[A_z(3)-A_z(4)\right] \\
  B_y(\tilde{1})&=\frac{1}{\Delta_x(4)}\left[A_z(4)-A_z(5)\right]
\end{align*}
And finally:
\begin{multline*}
  B_x(\tilde{0})\nu_x(\tilde{0})\tilde{\Delta}_x(\tilde{0})
  +B_y(\tilde{1})\nu_y(\tilde{1})\tilde{\Delta}_y(\tilde{1}) \\
  -B_x(\tilde{3})\nu_x(\tilde{3})\tilde{\Delta}_x(\tilde{3})
  -B_y(\tilde{0})\nu_y(\tilde{0})\tilde{\Delta}_y(\tilde{0}) = J(4)\cdot\tilde{S}(4)
\end{multline*}
Now the following coefficients are introduced:
\begin{align}
  \label{eq:16}
  g_y(1)&=\frac{\tilde{\Delta}_x(\tilde{0})}{\Delta_y(1)}\nu_x(1) &
  g_x(4)&=\frac{\tilde{\Delta}_y(\tilde{1})}{\Delta_x(4)}\nu_y(4) \\
  g_y(4)&=\frac{\tilde{\Delta}_x(\tilde{3})}{\Delta_y(4)}\nu_x(4) &
  g_x(3)&=\frac{\tilde{\Delta}_y(\tilde{0})}{\Delta_x(3)}\nu_y(3)
\end{align}
And now:
\begin{multline}
  \label{eq:17}
  \left[g_y(1)+g_x(4)+g_y(4)+g_x(3)\right] A_z(4) \\
  -g_y(1)A_z(1)-g_x(4)A_z(5)-g_y(4)A_z(7)-g_xA_z(3) = J(4)\tilde{S}(4)
\end{multline}
This equation can be expressed in matrix form as
\begin{gather}
  \label{eq:18}
  \matr{G}\vect{A}=\vect{J}\circ\vect{S}=\vect{\Theta}
\end{gather}
what is especially useful for larger grids.  The system matrix has a
special structure which makes it easy to directly generate it from the
grid (algorithm~\ref{alg:1}).\footnote{If you are familiar to the
  nodal analysis of electrical networks, you will probably notice that
  the rules are quite similar.}
\begin{algorithm}
  \ForAll{vertices $v$}{
    \ForAll{out edges $e$ of $v$}{
      $\tilde{e}=\text{secondary grid's crossing edge}$\;
      $l_{\text{sec}}=\operatorname{length}(\tilde{e})$\;
      $l_{\text{prim}}=\operatorname{length}(e)$\;
      $g=\nu\cdot l_{\text{sec}}/l_{\text{prim}}$\;
      %
      $\vect{G}(\operatorname{source}(e),\operatorname{target}(e))=-g$\;
      %
      $\vect{G}(\operatorname{source}(e),\operatorname{source}(e))+=g$\;
    }
  }
\caption{Generation of matrix entries for a grid's inner node.}
\label{alg:1}
\end{algorithm}


\section{Field Dependent Reluctivity}
\label{sec:field-depend-reluct}

Nonlinear material properties appear frequently in magnetic
calculations.  So it is nessesary to deal with them.  Usually an
iterative process is used to solve the nonlinear equations.

\subsection{Simple Iteration}
\label{sec:simple-iteration}

The simplest way to handle nonlinear material properties is to
iteratively resolve \eqref{eq:18} and adjust the reluctivity after each
step.  It is also possible to implement a damped variant where a
weighted average of the solution and previous solution is computed.
This algorithm has the following advantages and
disadvantages:\footnote{See \parencite[chapter 5]{Silvester:1996} for
  a more detailed discussion}
\begin{description}
\item[Advantage]
  \begin{itemize}
  \item Simple, no derivation required
  \end{itemize}
\item[Disadvantage]
  \begin{itemize}
  \item slow convergence
  \item no certain way for error estimation
  \end{itemize}
\end{description}


\subsection{\personname{Newton} Iteration}
\label{sec:newton-iteration}

The \personname{Newton} procedure \parencite{Wikipedia:Newton} is
described by the following equation: \index{Newton procedure|emph}
\begin{gather}
  \label{eq:21}
  \vect{A}(q+1)=\vect{A}(q)-\jacmatr^{-1}\cdot\vect{f}(\vect{A}(q))\text{~.}
\end{gather}
with
\begin{gather*}
  \vect{f}(\vect{A}(q))=\matr{G}\vect{A}(q)-\vect{\Theta}
\end{gather*}
and
\begin{align*}
  \jacmatr 
  & \quad\text{\personname{Jacobi} matrix of $\vect{f}(\vect{A})$,} \\
  \gls{q} & \quad\text{the \glsdesc{q}.}
\end{align*}
Of course we want to avoid computing the inverse of the
\personname{Jacobi} matrix \jacmatr{} -- a big sparse matrix -- due to its
computational costs.  So we reformulate \eqref{eq:21} as
\begin{gather}
  \label{eq:41}
  [\upDelta\vect{A}](q)=\vect{A}(q+1)-\vect{A}(q)\text{~,} \\
  \jacmatr\cdot[\upDelta\vect{A}](q)=-\vect{f}(\vect{A}(q))\text{~.}
\end{gather}
This system of equations can be solved for $[\upDelta\vect{A}](q)$ by some
suitable method like the conjugate gradient method.
\par To solve this system of equations we need the Jacobian matrix's
entries.  The \personname{Jacobi} matrix
\index{Jacobi matrix@\personname{Jacobi} matrix|emph} is defined as
\begin{gather}
  \label{eq:29}
  \jacmatr =
  \begin{pmatrix}
    \partdiff{f_0}{A_0} & \partdiff{f_0}{A_1} & \dots &
    \partdiff{f_0}{A_{\xmax{n}}} \\
    \partdiff{f_1}{A_0} & \partdiff{f_1}{A_1} & \dots &
    \partdiff{f_1}{A_{\xmax{n}}} \\
    \vdots & \vdots & \ddots & \vdots \\
    \partdiff{f_{\xmax{m}}}{A_0} & \partdiff{f_{\xmax{m}}}{A_1} &
    \dots & \partdiff{f_{\xmax{m}}}{A_{\xmax{n}}}
  \end{pmatrix}\text{~.}
\end{gather}
A single entry of this matrix is
\begin{gather}
  \label{eq:23}
  f_m=-\Theta_m+\sum_{k=0}^{\xmax{n}}g_{mk}A_k
\end{gather}
with the derivation
\begin{gather}
  \label{eq:24}
  \partdiff{f_m}{A_n}
  =\left(\sum_{k=0}^{\xmax{n}}\partdiff{}{A_n}g_{mk}A_k\right)
\end{gather}
Let's analyse this derivation in parts.
\begin{gather}
  \label{eq:26}
  \sum_{k=0}^{\xmax{n}}\partdiff{}{A_n}g_{mk}A_k
  =\sum_{k=0}^{\xmax{n}}\left(\partdiff{g_{mk}}{A_n}A_k
    +g_{mk}\partdiff{A_k}{A_n}\right)
\end{gather}
The second term evaluates to
\begin{gather}
  \label{eq:27}
  g_{mk}\partdiff{A_k}{A_n}=
  \begin{cases}
    g_{mk} & k = n \\
    0     & k \ne n
  \end{cases}
\end{gather}
and thereby only appears on the diagonal.  In the first term it is
nessesary to apply the chain rule of differentiation because $g_{mk}$
depends indirectly on $A_n$:
\begin{gather}
  \label{eq:28}
  \partdiff{g_{mk}}{A_n}A_k=\partdiff{g_{mk}}{B}\partdiff{B}{A_n}\cdot
  A_k=\pm\frac{\tilde{\Delta}}{\Delta^2}\partdiff{\nu}{B}\cdot A_k
\end{gather}
Despite this lengthy calculation the final result is rather simple.
First of all, the \personname{Jacobi} matrix has the same structure as
the system matrix \matr{G}:  Elements which are zero in \matr{G} are also zero
in \jacmatr.
Further it turns out that every edge -- beside the already introduced
coefficient $g$ -- has attached a jacobi coefficient $g_{\mathrm{J}}$ which is
\begin{gather}
  \label{eq:11}
  g_{\mathrm{J}}=g+\frac{\tilde{\Delta}}{\Delta}B\partdiff{\nu}{B}
\end{gather}
With this definition the \personname{Jacobi} matrix can be generated
in the same way as the system matrix (algorithm~\ref{alg:1}).

\subsubsection{Damped \personname{Newton} Procedure}
\label{sec:damp-newt-proc}

It is possible to introduce a \glsdesc{delta}~\gls{delta} into the
\personname{Newton} process:
\begin{gather}
  \label{eq:30}
  \vect{A}(q+1)=\vect{A}(q)+\delta\cdot[\upDelta\vect{A}](q)
\end{gather}
The damping factor is chosen so that
\begin{gather}
  \label{eq:31}
  \delta\le\min\left(1,
    \frac{\vectnorm{[\upDelta\vect{A}](q-1)}}{\vectnorm{[\upDelta\vect{A}](q)}}\right)\text{~.}
\end{gather}
So, if the error becomes bigger in the current step than in the last
step then the damping factor is reduced.

\section{Boundary Conditions}
\label{sec:bound-treatm-cond}

\subsection{\personname{Neumann} Boundary Condition}
\label{sec:neumann}

The \personname{Neumann} condition is about setting the first
derivative to a prescribed value.  In the case considered here, this
means nothing else then presetting the magnetic flux density parallel
to the boundary.
\begin{figure}
  \centering
  \includeinkscape{Neumann-Boundary}
  \caption{Boundary with a \personname{Neumann} condition}
  \label{fig:neumann-boundary}
\end{figure}
Figure~\ref{fig:neumann-boundary} shows how the \personname{Neumann}
condition is applied in the context of finite integrals.  The
integration path around the boundary vertex is in parts a ``ghost'' path
because it lies outside the computation domain.  It is choosen in such
a way that the surounded vertex is at its center.  \par The relevant
equations for a \personname{Neumann} vertex are:
\begin{align*}
  B_x(3)&=\frac{1}{\Delta_y(3)}\left[A_z(6)-A_z(3)\right]  \\
  B_y(3)&=\frac{1}{\Delta_x(3)}\left[A_z(3)-A_z(4)\right]  \\
  B_x(0)&=\frac{1}{\Delta_y(0)}\left[A_z(3)-A_z(0)\right]
\end{align*}
And finally:
\begin{gather}
  \label{eq:19}
  B_y(3)g_x(3)-B_x(3)g_y(3)+B_x(0)g_y(0)
    =J_z(3)\tilde{S}(3)+B_{\text{out}}\nu_{\text{out}}l_{\text{out}}
\end{gather}
So, the \personname{Neumann} boundary condition modifies the
equation's right hand side.  In the case of the homogenious
\personname{Neumann} boundary condition
($B_{\text{out}}=\qty{0}{\tesla}$) it is the same equation as for a
grid's inner vertex.  That means the homogenious \personname{Neumann}
condition arises automatically if no other condition is set.
\emph{Important:}  The sign of the right hand side's modification term
depends onto the boundary:
\begin{gather*}
  \text{sign}=
  \begin{cases}
    -1 & \text{the boundary is parallel to the axis} \\
    +1 & \text{the boundary is \emph{anti}parallel to the axis}
  \end{cases}
\end{gather*}

\subsection{\personname{Dirichlet} Boundary Condition}
\label{sec:dirichlet}

The \personname{Dirichlet} boundary condition describes the case that no flux
crosses the boundary.  In other words the flux perpendicular to the
boundary is zero or -- expessed in terms of the magnetic vector
potential -- the magnetic vector potential along the boundary has a
constant prescribed value (figure~\ref{fig:diri-condition}).
\begin{figure}
  \centering
  \includeinkscape{Dirichlet-Boundary}
  \caption{Boundary with a \personname{Dirichlet} condition}
  \label{fig:diri-condition}
\end{figure}
 Because the potential is known in advance
the vertice's equation can be eleminated from \eqref{eq:18}:
\begin{multline}
  \left[g_x(n)+g_y(n)+g_x(n-1)+g_y(n-\xmax{m})\right]A_z(n) \\
  -g_x(n)A_z(n+1)-g_y(n)A_z(n+\xmax{m})-g_y(n-\xmax{m})A_z(n-\xmax{m}) \\
  = J_n\tilde{S}_n+g_x(n-1)A_z(n-1)
\end{multline}
This leads to algorithm~\ref{alg:2} for adding \personname{Dirichlet}
vertices to the equation system.
\begin{algorithm}
  \ForAll{$\operatorname{out\_edges}(\text{\personname{Dirichlet}
      vertex})$}{
    Add $g\cdot A_z$ to the equation's right hand side of the target
    vertex\;
  }
  \caption{Add \personname{Dirichlet} vertices to the equation system}
  \label{alg:2}
\end{algorithm}

\section{Coupled Boundaries}
\label{sec:coupled-boundaries}

With finite integrals the implementations of boundary coupling is so
natural that it is almost not worth to mention it at all.
\begin{figure}
  \centering
  \includeinkscape{Interface-polar-cartesian}
  \caption{Interface between a cartesian and polar grid patch}
  \label{fig:polar-cartesian}
\end{figure}
Figure~\ref{fig:polar-cartesian} shows how two grids patches can be
coupled.  Only the calculation of the integration path'es length and
the surrounded surface area must be adjusted.  
% \par Coming soon: Antiperiodic coupling.


%%% Local Variables: 
%%% mode: latex
%%% TeX-master: "manual"
%%% ispell-local-dictionary: "en_US"
%%% End: 

% -*- coding: utf-8 -*-
%
% Copyright © 2012-2016 Philipp Büttgenbach
%
% This file is part of ulphi, a CAE
% tool and library for computing electromagnetic fields.
%
% Permission is granted to copy, distribute and/or modify this
% document under the terms of the \gls{gnu} Free Documentation
% License, Version 1.3 or any later version published by the Free
% Software Foundation; with no Invariant Sections, no Front-Cover
% Texts, and no Back-Cover Texts.  A copy of the license is included
% in the section entitled GNU Free Documentation License.
%
% This manual is distributed in the hope that it will be useful, but
% WITHOUT ANY WARRANTY; without even the implied warranty of
% MERCHANTABILITY or FITNESS FOR A PARTICULAR PURPOSE.
%

\chapter{Linear \personname{Poisson} Equation}

\section{Electrostatics}
\label{sec:electrostatics}

The \personname{Poisson} equation reads:
\begin{align}
  \laplaceop \phi &= \frac{\rho}{\epsilon} \\
  \Leftrightarrow\quad\divop \grad \phi &= \frac{\rho}{\epsilon}
\end{align}
This equation is integrated by the theorem of \personname{Gauß}:
\begin{gather}
  \epsilon\oiinteg{S}{\grad \phi}{S} = \iiinteg{V}{\rho}{V} = Q
  \label{eq:20}
\end{gather}
The $\grad$-operator is kept and not replaced by an integral
expression.  This is alright because keeping this operator doesn't
introduce any averaging of grid edges so that the advantages of using
\gls{fit} pointed out at the begining of
chapter~\ref{cha:discretization} are preserved.
\begin{figure}
  \centering
  \input{../images/Poisson-grid-3D.tex}
  \caption{A basic grid cell in 3D}
  \label{fig:BasicGridCellPoisson3D}
\end{figure}
Figure~\ref{fig:BasicGridCellPoisson3D} shows a basic grid cell.
Aplying equation~\eqref{eq:20} to this grid leads to:
\newcommand*{\phiepsilonSl}[2]{
  \ensuremath{\phi_{#1,#2}
  \frac{\epsilon_{#1,#2}\cdot S_{#1,#2}}{l_{#1,#2}}}}
\begin{multline}
  +\phiepsilonSl{13}{14}+\phiepsilonSl{13}{16} \\
  -\phiepsilonSl{12}{13}-\phiepsilonSl{10}{13} \\
  +\phiepsilonSl{13}{22}-\phiepsilonSl{4}{13} = Q
\end{multline}
with
\begin{align*}
  l&\quad \text{the grid edge'es length} \\
  S&\quad \text{Surface perpendicular to this edge}
\end{align*}
Doing so for all cells of a larger grid yields an equation system
which has the same structure as \eqref{eq:18}:
\begin{gather}
  \matr{G}\vect{\phi}=\vect{\rho}\circ\vect{V}=\vect{Q}
\end{gather}
with
\begin{gather}
  \label{eq:32}
  g=\epsilon\frac{S}{l}
\end{gather}

\section{Reduction to 2D}
\label{sec:reduction-2d}

\begin{itemize}
\item We asume the homogeneous \personname{Neumann} condition on surface
  $S_{13,22}$ and $S_{4,13}$.
\item The grid shall have the fixed length $l_D$ in $z$-direction.
\end{itemize}
It follows that the source term is a surface charge
density~\gls{sigma} and the coefficients forming the system matrix are
\begin{gather}
  \label{eq:22}
  g=\frac{\tilde{\Delta}}{\Delta}\gls{epsilon}\text{~.}
\end{gather}

\section{Thermostatics}
\label{sec:thermostatics}

\begin{gather}
  \label{eq:33}
  \laplaceop T = -\frac{1}{\lambda}\partdiff{\gls{Phith}}{V}
\end{gather}
\begin{itemize}
\item The source term is \gls{qth}
\item $g=\frac{\tilde{\Delta}}{\Delta}\gls{lambdath}$
\end{itemize}
  
%%% Local Variables: 
%%% mode: latex
%%% TeX-master: "manual"
%%% ispell-local-dictionary: "en_US"
%%% End: 

% -*- coding: utf-8 -*-
%
% Copyright © 2012-2016 Philipp Büttgenbach
%
% This file is
% part of ulphi, a CAE tool and library for computing electromagnetic
% fields.
%
% Permission is granted to copy, distribute and/or modify this
% document under the terms of the \gls{gnu} Free Documentation
% License, Version 1.3 or any later version published by the Free
% Software Foundation; with no Invariant Sections, no Front-Cover
% Texts, and no Back-Cover Texts.  A copy of the license is included
% in the section entitled GNU Free Documentation License.
%
% This manual is distributed in the hope that it will be useful, but
% WITHOUT ANY WARRANTY; without even the implied warranty of
% MERCHANTABILITY or FITNESS FOR A PARTICULAR PURPOSE.
%

\chapter[Important techniques in electromagnetic modelling]{
Important techniques in electromagnetic modelling of electrical
machines}

\label{cha:import-techn-electr}


\section{Modelling blocks of laminated material}
\label{cha:magnetic-resistance}
\index{Laminated material!modelling|emph}

When modelling blocks of laminated material, it is very impractical to
impossible to model each layer separately: Standard thicknesses for
electrical steel are in the range of 0.2~millimetre to
1~millimetre but in some applications even thinner sheets
are used.  Therefore laminated materials are modelled as a homogeneous
material which mimics the behaviour of the laminated material.  In the
following the properties for such an homogenized material are derived
using a magnetic circuit model of the laminated material.\footnote{This topic
is also treated in i. e. \cite[][Section 7.4]{Salon:1995}.}


\subsection{Flux passing throw a cubic block}

\subsubsection{Flux and lamination direction are parallel}
\label{sec:flux-lamin-direct}

Figure~\ref{fig:Laminated-Block-Flux-Parallel-Geometry} shows the
case where the magnetic flux is passing a laminated block in
lamination direction.  The corresponding magnetic circuit is shown in
figure~\ref{fig:Laminated-Block-Flux-Parallel-Circuit}.
\begin{figure}
  \centering
  \subfloat[Geometry]{
    \input{../images/Laminated-Block-Flux-parallel-axonometric.pdf_tex}
    \label{fig:Laminated-Block-Flux-Parallel-Geometry}
  }
  \hfil
  \subfloat[equivalent magnetic circuit]{
    \input{../images/Laminated-Block-Flux-parallel-equiv-circ.pdf_tex}
    \label{fig:Laminated-Block-Flux-Parallel-Circuit}
  }
  \caption{Magnetic flux passing a laminated block in lamination direction}
  \label{fig:Lam-Mat-Flux-parallel}
\end{figure}
\label{sec:cart-coord}
In general, the \glsdesc{Rm}~$\Rm$ of a cubic block can be expressed as
\begin{gather}
  \Rm=\frac{l\nu}{S}
\end{gather}
with
\begin{align*}
  l&\quad\text{the section's length}\text{~,}\\
  S&\quad\text{the section's cross section}\text{~,}\\
  \gls{nu}&\quad\text{the section's \glsdesc{nu}}\text{~.}
\end{align*}
\index{Magnetic resistance} For the magnetic resistors in the circuit
(figure~\ref{fig:Laminated-Block-Flux-Parallel-Circuit}) we find:
\begin{align}
  \RmAir=\frac{l\cdot\nuAir}{\SAir}\text{~,} &&
  \RmFe=\frac{l\cdot\nuFe}{\SFe}
\end{align}
with
\begin{align}
  \SFe=a\cdot \kFe \dsheet\text{~,} &&
  \SAir=a\cdot (1-\kFe) \dsheet\text{~.}
\end{align}
In these equations $\kFe$ is the stacking factor or iron fill factor.
The total magnetic resistance of these two parallel resistors is
\begin{gather}
  \label{eq:15}
\begin{split}
  \Rm&=\frac{\RmFe\cdot\RmAir}{\RmFe + \RmAir} \\
  &=\frac{l}{a\dsheet}
    \frac{1}{\frac{\kFe}{\nuFe}+\frac{1-\kFe}{\nuAir}}\text{~.}
\end{split}
\end{gather}
From equation~\eqref{eq:15} follows the homogenized reluctivty if the
flux travels the block parallel to the lamination direction as
\begin{gather}
  \nueffpar=\frac{1}{\frac{\kFe}{\nuFe}+\frac{1-\kFe}{\nuAir}}\text{~.}
\end{gather}


\subsubsection{Flux and lamination direction are perpendicular}
\label{sec:perp-lamin-direct}

Figure~\ref{fig:Lam-Mat-Flux-perpendicular} shows the geometry and
equivalent magnetic circuit for the case the magnetic flux is passing
the laminated block parallel to the lamination direction.
\begin{figure}
  \centering
  \subfloat[Geometry]{
    \input{../images/Laminated-Block-Flux-perpendicular-axonometric.pdf_tex}}
  \hfil
  \subfloat[equivalent magnetic circuit]{
    \input{../images/Laminated-Block-Flux-perpendicular-equiv-circ.pdf_tex}
  }
  \caption{Magnetic flux passing a laminated block perpendicular to lamination direction}
  \label{fig:Lam-Mat-Flux-perpendicular}
\end{figure}
The magnetic resistances in the equivalent magnetic circuit turn out to be
\begin{gather}
  \label{eq:25}
  \Rm=\frac{d\nu}{S}
\end{gather}
with
\begin{align*}
  d&\quad\text{the section's thickness}\text{~,}\\
  S&\quad\text{the section's cross section}\text{~,}\\
  \gls{nu}&\quad\text{the section's \glsdesc{nu}}\text{~.}
\end{align*}
Especially it is
\begin{align}
  \RmFe=\frac{\dsheet}{S}\cdot\kFe\nuFe\text{~,} &&
  \RmAir=\frac{\dsheet}{S}\cdot(1-\kFe)\nuAir\text{~.}
\end{align}
And the equivalent circuit's total magnetic resistance is
\begin{gather}
  \label{eq:6}
  \begin{split}
    \Rm&=\RmFe + \RmAir \\
    &=\frac{\dsheet}{S}\cdot\left[\kFe\nuFe+(1-\kFe)\nuAir\right]\text{~.}
  \end{split}
\end{gather}
From equation~\eqref{eq:6} follows the reluctivity seen by the
magnetic flux travelling the laminated block perpendicular to the
lamination direction as
\begin{gather}
  \nueffperp= \kFe\nuFe+(1-\kFe)\nuAir \text{~.}
\end{gather}


\subsection{Flux passing through a toroidal Block}
\label{sec:cylinder-coordinates}

\begin{figure}
  \centering
  \input{../images/Laminated-Block-Toroidal.pdf_tex}
  \caption{Toroidal laminated block}
  \label{fig:Toroidal-laminated-block}
\end{figure}
First, we want to derive the magnetic resistance passing the block in
figure~\ref{fig:Toroidal-laminated-block} in radial direction:
\begin{gather}
  \Rm=\frac{\Um}{\Phi}\text{~,} \\
  \Um=Hr(\alpha_2-\alpha_1)\text{~,} \\
  \Phi=\iinteg{S}{\vect{B}}{\vect{S}}
  =\frac{h\Um}{\alpha_2-\alpha_1}
  \integg{\rin}{\rout}{\frac{\mu(r)}{r}}{r}\text{~.}
  \intertext{It follows:}
  \label{eq:78}
  \Rm=\frac{\alpha_2-\alpha_1}{h}\left(
    \integg{\rin}{\rout}{\frac{\mu(r)}{r}}{r}\right)^{-1}\text{~.}
\end{gather}
When passing this block in angular direction, the magnetic flux
experiences the following resistance:
\begin{gather}
  \Rm=\frac{\Um}{\Phi}\text{~,} \\
  B=\frac{\Phi}{hr(\alpha_2-\alpha_1)}\text{~,} \\
  \Um =\integ{}{\vect{H}}{\vect{l}}
  =\frac{\Phi}{h(\alpha_2-\alpha_1)}
  \integg{\rin}{\rout}{\frac{1}{r\cdot\mu(r)}}{r}\text{~.}
  \intertext{It follows:}
  \label{eq:77}
  \Rm=\frac{1}{h(\alpha_2-\alpha_1)}
  \integg{\rin}{\rout}{\frac{1}{r\cdot\mu(r)}}{r}\text{~.}
\end{gather}
Equation \eqref{eq:78} and \eqref{eq:77} give the exact solutions for
the case the magnetic flux is passing a toroidal block in or
perpendicular to the lamination direction.  This solution is not a
very practical solution:  Because the material's permeability depends
on the magnetic flux density which is inhomogeniously distributed, the
material must also be seen as an inhomogenious material.  Therefore it
is impossible to take the material's permeability $\mu(r)$ out
of the integral.  But what are we really interested in?  We're
interested in cases where \rin and \rout differ only
by the sheet's thickness; that means cases where
$\rin\approx\rout$.  In these cases we can assume,
that the magnetic flux density~$B$ and therefore also the
permeability~$\mu$ is almost homogeneous in the computation domain.
With this precondition it follows, that the permeability is the same
as in the Cartesian coordinate case.  When being in doubt that this
precondition is valid it is of course possible to select a higher
number of vertices in radial direction.


\section{Nonlinear timeharmonic computations}
\label{sec:nonl-timeh-comp}
\index{Computation!nonlinear timeharmonic|emph}

A timeharmonic computation implies that all quantities have a
sinusoidal waveform.  On the other hand material non-linearities
introduce a notable distortion.  This is shown in
figure~\ref{fig:waveform-distortion} for a typical electrical steel
\footnote{The material used as an example throughout this section is a
non oriented and fully processed electrical steel: M330-35A.  The
material data is taken from \cite{Cogent:NOFP}.}.
\begin{figure}
  \begin{maxipage}
  \centering
  \subfloat[excitation by a sinusoidal current]{
    \begin{tikzpicture}
      \begin{axis}[
        height=.44\textheight,
        tick style={line width=.18mm},
        view={132}{30},
        domain=0:pi,
        xlabel=$\frac{\omega t}{\pi}$,
        ylabel=$H\cdot\frac{\metre}{\kilo\ampere}$,
        zlabel=${B}/{\tesla}$,
        line width=.35mm]
        \addplot3[mark=none,line width=.7mm,style=dotted] file {../datasets/M330-35-A.csv};
        \addplot3[mark=none,line width=.7mm,style=loosely dashdotted]
        ({y/pi}, {.16*sin(deg(y))}, {0});
        \addplot3[mark=none,line width=.7mm,style=densely dashdotted]
        ({y/pi}, {2.8*sin(deg(y))}, {0});
        \addplot3[mark=none,line width=.7mm,style=loosely dashdotted]
        file {../datasets/M330-35-A_H-Ampl-160.csv};
        \addplot3[mark=none,line width=.7mm,style=densely dashdotted]
        file {../datasets/M330-35-A_H-Ampl-2800.csv};
      \end{axis}
    \end{tikzpicture}
  }\\[\fill]
  \subfloat[excitation by a sinusoidal voltage]{
    \begin{tikzpicture}
      \begin{axis}[
        height=.44\textheight,
        tick style={line width=.35mm},
        view={132}{30},
        domain=0:pi,
        xlabel=$\frac{\omega t}{\pi}$,
        ylabel=$H\cdot\frac{\metre}{\kilo\ampere}$,
        zlabel=${B}/{\tesla}$,
        line width=.35mm]
        \addplot3[mark=none,line width=.7mm,style=dotted] 
        file {../datasets/M330-35-A.csv};
        \addplot3[mark=none,line width=.7mm,style=loosely dashdotted]
        file {../datasets/M330-35-A_B-Ampl-0.6.csv};
        \addplot3[mark=none,line width=.7mm,style=densely dashdotted]
        file {../datasets/M330-35-A_B-Ampl-1.6.csv};
        \addplot3[mark=none,line width=.7mm,style=loosely dashdotted]
        ({y/pi}, {0}, {.6*sin(deg(y))});
        \addplot3[mark=none,line width=.7mm,style=densely dashdotted]
        ({y/pi}, {0}, {1.6*sin(deg(y))});
      \end{axis}
    \end{tikzpicture}
  }
  \caption{Waveform distortion due to nonlinear material properties}
  \label{fig:waveform-distortion}
  \end{maxipage}
\end{figure}
In these figures the $H$-$B$-plane shows the original
magnetization curve and the other planes show one of the quantities
$B$ and $H$ over time when they are mirrored at the
magnetization curve.  \par So, how can despite these non-linearities a
timeharmonic computation be done?  In the presence of non-linearities
the goal of a timeharmonic computation isn't anymore to match the
exact waveform -- in the presence of non-linearities the goal is to
match the distorted waveform's fundamental.  To reach this goal the
magnetization curve is transformed.  First we observe that in the
case of sinusoidal quantities:
\begin{align}
  \label{eq:63}
  B&\sim U &&
  \text{because}&&\Phi=\integ{}{U}{t}
  \intertext{and} %%
  H&\sim I && \text{because}&&H\cdot l=\Theta
\end{align}
From that it follows that depending on whether the system is excited
by a sinusoidal current or a sinusoidal voltage the magnetization
curve must be transformed in a different way.  In the case of
excitation by a sinusoidal current the transformation is
carried out in the following way:
\begin{enumerate}
\item We assume that we start with values measured by the material
  supplier.  Typically the peak field intensity and the peak
  polarization are provided.
\item As stated in equation~\eqref{eq:63} the flux density and not the
  polarization is required:
  \begin{gather}
    \label{eq:64}
    B = B_{\mathrm{i}}+\mu_0 H
  \end{gather}
\item The resulting $(H;B)$-value-pairs are interpolated by a
  cubic spline (or some other suitable interpolating function).
\item From the provided field intensity interval a set of discrete
  values is chosen.  For each of these values $\hat{H}$ we compute:
  \begin{enumerate}
  \item $H(\omega t)=\hat{H}cos(\omega t)$ with $\omega
    t\in[0;\pi[$
  \item $B(\omega t)=B(H(\omega t))$
  \item Find the fundamental~$\hat{B}$ of $B(\omega t)$ by
    using the Fourier transformation
  \item $(\hat{H};\hat{B})$ is one value pair of the
    transformed curve
  \end{enumerate}
\end{enumerate}
In the case of excitation by a sinusoidal voltage $H$ and $B$
must be interchanged.  Figure~\ref{fig:Transformed-mag-curves} shows
the result of this procedure for the material M330-35-A.
\begin{figure}
  \begin{maxipage}
  \centering
  \subfloat[transformed curve for current input]{
    \begin{tikzpicture}
      \begin{semilogxaxis}[
%        height=.44\textheight,
%        width=.7119349550499537\textheight,
        width=.8\textwidth,
        height=.49442719099991583\textwidth,
        xlabel=$H\cdot\frac{\metre}{\kilo\ampere}\quad\longrightarrow$,
        ylabel=${B}/{\tesla}\quad\longrightarrow$,
        grid=major,
        grid style={line width=.18mm},
        tick style={line width=.35mm},
        legend style={anchor=south east,at={(.92,.08)}},
        line width=.35mm]
        \addplot[mark=none,line width=.7mm,style=dotted]
        file {../datasets/M330-35-A_xy.csv};
        \addplot[mark=none,line width=.7mm,style=loosely dashed]
        file {../datasets/M330-35-A_current-input.csv};
        \legend{orig. curve, transformed}
      \end{semilogxaxis}
    \end{tikzpicture}
  } \\[\fill]
  \subfloat[transformed curve for voltage input]{
    \begin{tikzpicture}
      \begin{semilogxaxis}[
%        height=.44\textheight,
%        width=.7119349550499537\textheight,
        width=.8\textwidth,
        height=.49442719099991583\textwidth,
        xlabel=$H\cdot\frac{\metre}{\kilo\ampere}\quad\longrightarrow$,
        ylabel=${B}/{\tesla}\quad\longrightarrow$,
        grid=major,
        grid style={line width=.18mm},
        tick style={line width=.35mm},
        legend style={anchor=south east,at={(.92,.08)}},
        line width=.35mm]
        \addplot[mark=none,line width=.7mm,style=dotted]
        file {../datasets/M330-35-A_xy.csv};
        \addplot[mark=none,line width=.7mm,style=loosely dashed]
        file {../datasets/M330-35-A_voltage-input.csv};
        \legend{orig. curve, transformed}
      \end{semilogxaxis}
    \end{tikzpicture}
}
  \caption{Transformed magnetization curves}
  \label{fig:Transformed-mag-curves}
  \end{maxipage}
\end{figure}

%%% Local Variables: 
%%% mode: latex
%%% TeX-master: "manual"
%%% ispell-local-dictionary: "en_US"
%%% End: 

% -*- coding: utf-8 -*-
%
% Copyright © 2012-2016 Philipp Büttgenbach
%
% This file is part of ulphi, a CAE tool and
% library for computing electromagnetic fields.
%
% Permission is granted to copy, distribute and/or modify this
% document under the terms of the \gls{gnu} Free Documentation
% License, Version 1.3 or any later version published by the Free
% Software Foundation; with no Invariant Sections, no Front-Cover
% Texts, and no Back-Cover Texts.  A copy of the license is included
% in the section entitled GNU Free Documentation License.
%
% This manual is distributed in the hope that it will be useful, but
% WITHOUT ANY WARRANTY; without even the implied warranty of
% MERCHANTABILITY or FITNESS FOR A PARTICULAR PURPOSE.
%

\chapter{Using Ulphi}
\label{cha:practical-aspects}

\section{Preprocessing}
\label{sec:preprocessing}

\subsection{Grids}
\label{sec:grids}

\begin{figure}
  \centering
    \hfil
    \subcaptionbox{cartesian}{
      \includeinkscape{Cart-Grid-Init}}
    \hfil
    \subcaptionbox{polar}{
      \includeinkscape{Polar-Grid-Init}}
    \hfil
  \caption{Supported grid types}
  \label{fig:sup-grid-types}
\end{figure}
Currently, two types of grids are supported (figure~\ref{fig:sup-grid-types}):
\begin{itemize}
\item Cartesian grids
\item Polar grids
\end{itemize}
Some notes about grids:
\begin{itemize}
\item Grids are defined by their \gls{llc} and their \gls{urc}.
\item At creation time the global coordinate system becomes the grids
  local coordinate system.
\item Only rotating and shifting the grid make that global and local
  coordinate system differ.
\item Grids are default initialized by a material of no conductivity and the
  free space permeability.
\item Material definitions inside a grid may overlap. Valid is the
  last definition.
\end{itemize}

\subsection{Grid Boundaries}
\label{sec:grid-boundaries}

A grid's boundaries are named as shown in figure~\ref{fig:boundary-names}.
\begin{figure}
  \begin{minipage}{.485\linewidth}
    \centering
    \includeinkscape{boundary-names}
    \caption{A grid's boundary names}
    \label{fig:boundary-names}
  \end{minipage}
  \hfil
  \begin{minipage}{.485\linewidth}
    \centering
    \includeinkscape{edge-orientation}
    \caption{Vertice numbering onto boundaries}
    \label{fig:Vertice-numbering-onto-boundaries}
  \end{minipage}
\end{figure}
Vertices onto the boundary have their own numbering as shown in
figure~\ref{fig:Vertice-numbering-onto-boundaries}.


\subsection{Grid Coupling}
\label{sec:grid-coupling}

When coupling boundaries some care must be taken by the user that it
is done correctly because the software can not detect all possible
error conditions:
\begin{itemize}
\item For a grid region at maximum two edges may be used as source
  edges in couplings.
\item These links must be defined on parallel edges.
\end{itemize}
After linking two edges the source edge will be hidden by the target
edge.  Figure~\ref{fig:GridCouplings} shows some examples.
\begin{figure}
  \centering
  \subcaptionbox{Correct forwarding for the corner vertex}{
    \includeinkscape{edge-linking-1}} \hfil
  \subcaptionbox{Correct forwarding for the corner vertex}{
    \includeinkscape{edge-linking-2}} \\[1.4\bigskipamount]
  \subcaptionbox{Wrong! The corner vertex exists twice}{
    \includeinkscape{edge-linking-3}}
  \caption{Examples for grid couplings}
  \label{fig:GridCouplings}
\end{figure}
Cyclic coupling of four grids is currently supported and correctly
handled when it is detected.  This support may be removed in the
future because it is a huge complication. So, avoid cyclic coupling of
four grids!


\subsection{Vertex Distributions}

The distribution of vertices inside a grid is defined by vertex
distributions.  Two types are currently implemented:
\begin{itemize}
\item A linear vertex distribution and
\item an exponential vertex distribution.
\end{itemize}
If you have special demands you can implement your own vertex
distribution type by deriving from the \texttt{VertexDistribution}
base class.


\subsection{Shapes}

Shapes are used to define materials and current carrying regions
inside a grid.  The following shapes are implemented:
\begin{description}
\item[Rectangle] A rectangle is a shape with four edges parallel to
  the coordinate system's axes.  This implies that in a polar system a
  rectangle has two rounded edges.
\item[Circle] is defined by its center point and its radius.
\end{description}
If you have special demands you can implement your own shape types by
deriving from the \texttt{Shape} base class.


\subsection{Reluctivity Models}
\label{sec:material-models}

Several reluctivity models are supported which are described in the
following.
\begin{description}
\item[Constant reluctivity] The constant reluctivity model is simply a
  constant value. An example for an material with a constant
  reluctivity is air.
\item[Arctan model] This model has its origin in the FLUX® \gls{cad}
  package \parencite{Cedrat:2006}. It uses an arctan function to model the
  nonlinear behaviour of materials like iron.
  \begin{gather}
    \label{eq:39}
    B(H)=\mu_0H+\frac{2B_{\mathrm{i,s}}}{{\pi}}\arctan\left(
    \frac{{\pi}(\mu_{\mathrm{r}}-1)\mu_0H}{2B_{\mathrm{i,s}}}\right)
  \end{gather}
  with
  \begin{align*}
    \mu_{\mathrm{r}}&\quad\text{rel. permeability at the origin} \\
    B_{\mathrm{i,s}}&\quad\text{saturation polarization}
  \end{align*}
  This model may be initialized from these two parameters or from
  measured data.  In the last case the
  \personname{Levenberg}-\personname{Marquardt} procedure is used to
  find a set of good parameters.  \par Although using such a model
  might look like a good idea on a first glance because a strictly
  monotone curve helps the \personname{Newton} process to converge, this model
  has some drawbacks:
  \begin{itemize}
  \item The $B$-$H$-characteristic is modeled but for solving equation
    \eqref{eq:41} the $\nu$-$B$-characteristic is required.  Because
    there is no analytical expression for the
    $\nu$-$B$-char\-ac\-ter\-is\-tic resulting from this model,
    \personname{Newton}'s procedure must be used.
  \item The calculation of the derivative $\partial\nu/\partial B$
    yields a lengthy expression which is hard to check for
    correctness.
  \item Most of the time real world magnetization curves are not
    fitted that well by this model.
  \end{itemize}
\item[Square root model] This model uses a square root function to
  approximate the $B$-$H$-curve:
  \begin{align}
    \label{eq:42}
    B(H) &=
    \mu_0H+B_{\mathrm{i,s}}\frac{H_a+1-\sqrt{(H_a+1)^2-4H_a(1-a)}}{2(1-a)}
    \intertext{with}
    H_a&=\mu_0H\frac{\mu_r-1}{B_{\mathrm{i,s}}}
  \end{align}
  In all other aspects this model is similar to the previous model.
\item[Spline model] This model uses a spline to interpolate measured
  values.  If extrapolation is required, a tangent line through the
  first or last data point is used.  This model is recommended for
  everyday use.
\end{description}
Figure~\ref{fig:reluct-model-perf} shows how these models perform with
real world data.
\begin{figure}
  \centering
  \sisetup{per-mode=fraction}
  \subcaptionbox{Arctan model}{
    \begin{tikzpicture}
      \begin{semilogyaxis}[
        height=.275\textheight,
        width=.44495934690622113\textheight,
        xlabel=$B/\unit{\tesla}\quad\longrightarrow$,
        ylabel=$\nu\cdot\unit{\henry\per\metre}\quad\longrightarrow$,
        grid=major,
        grid style={line width=.18mm},
        tick style={line width=.35mm},
        legend style={at={(0.04,0.96)},anchor=north west},
        line width=.35mm]
        \addplot[mark=diamond,only marks]
          table[x=B, y=NuOrig] {M400_50A-arctan.csv};
        \addplot[smooth,mark=none,line width=.7mm,style=dotted]
          table[x=B, y=NuModel] {M400_50A-arctan.csv};
        \legend{orig. data, arctan model}
      \end{semilogyaxis}
    \end{tikzpicture}
  }\\ \bigskip
  \subcaptionbox{Square root model}{
    \begin{tikzpicture}
      \begin{semilogyaxis}[
        height=.275\textheight,
        width=.44495934690622113\textheight,
        xlabel=$B/\unit{\tesla}\quad\longrightarrow$,
        ylabel=$\nu\cdot\unit{\henry\per\metre}\quad\longrightarrow$,
        grid=major,
        grid style={line width=.18mm},
        tick style={line width=.35mm},
        legend style={at={(0.04,0.96)},anchor=north west},
        line width=.35mm]
        \addplot[mark=diamond,only marks]
          table[x=B, y=NuOrig] {M400_50A-sqrt.csv};
        \addplot[smooth,mark=none,line width=.7mm,style=dotted]
          table[x=B, y=NuModel] {M400_50A-sqrt.csv};
        \legend{orig. data, sqrt model}
      \end{semilogyaxis}
    \end{tikzpicture}
  }\\ \bigskip
  \subcaptionbox{Spline model}{
    \begin{tikzpicture}
      \begin{semilogyaxis}[
        height=.275\textheight,
        width=.44495934690622113\textheight,
        xlabel=$B/\unit{\tesla}\quad\longrightarrow$,
        ylabel=$\nu\cdot\unit{\henry\per\metre}\quad\longrightarrow$,
        grid=major,
        grid style={line width=.18mm},
        tick style={line width=.35mm},
        legend style={at={(0.04,0.96)},anchor=north west},
        line width=.35mm]
        \addplot[mark=diamond,only marks]
          table[x=B, y=NuOrig] {M400_50A-spline.csv};
        \addplot[smooth,mark=none,line width=.7mm,style=dotted]
          table[x=B, y=NuModel] {M400_50A-spline.csv};
        \legend{orig. data, spline model}
      \end{semilogyaxis}
    \end{tikzpicture}
  }
  \caption[Reluctivity models and how they perform with real world
  data]{
    Reluctivity models and how they perform with real world
    data. The material is M400-50A.}
  \label{fig:reluct-model-perf}
\end{figure}


\subsection{Material and Load Region Boundaries Inside a Grid}

Inside a grid the boundaries of material and load regions are
approximated in order to match the grid
(figure~\ref{fig:ApproxShapes}).
\begin{figure}
  \centering
  \includeinkscape{Approx-Shapes}
  \caption{Geometry approximation inside grids}
  \label{fig:ApproxShapes}
\end{figure}
Errors due to this approximation can be kept small by using a
reasonable fine grid.  If you want to suppress this approximation, you
should carefully line up your geometry with the secondary grid.


\section{Solving}
\label{sec:solving}

There are three different solving procedures implemented:
\begin{description}
\item[Linear solver] This solver is used if all material properties
  are field independent.
\item[Iterative solver] This is a simple iterative solver.  Good luck!
\item[\personname{Newton} iteration] This solver uses the
  sophisticated \personname{Newton} procedure and is the best choice
  for nonlinear tasks (field dependent relucitivty).
\end{description}


\section{Post processing}
\label{sec:postprocessing}

From the grid graph the vertice's positions and potentials can be
extracted.
In the python interface this is done by:
\\
\begin{tabular}{lp{.82\textwidth}}
\pythonlogo & \begin{lstlisting}[language=python]
# ... initialize grid ...
grid_compiler = GridCompiler(grid)
grid_graph = grid_compiler()
grid_graph.Newton(1e-5, 24)

# Access results
grid_graph.positions
grid_graph.potentials
\end{lstlisting}
\end{tabular}


\subsection{Using a Path}
\label{sec:using-path}

\begin{figure}
  \centering
  \hfil
  \subcaptionbox{evaluate flux}{
    \includeinkscape{calc-flux-with-path}}
  \hfil
  \subcaptionbox{evaluate magnetic voltage}{
    \includeinkscape{calc-magnetic-voltage-with-path}}
  \hfil
  \caption{Using a path for postprocessing}
  \label{fig:UsePath}
\end{figure}
To evaluate the total flux\footnote{The total flux can be computed
from the average flux density.} and the magnetic voltage, a path is
used.
\\
\begin{tabular}{lp{.82\textwidth}}
\pythonlogo & \begin{lstlisting}[language=python]
# Initialize a path
path = PathInGridGraph(grid_graph, pointlist)

# Access average flux density
path.average_flux_density

# Access the magnetic voltage
path.magnetic_voltage
\end{lstlisting}
\end{tabular}


\subsection{Check Your Results}

It is always important to check the results:
\begin{itemize}
\item Does the field line plot look reasonable?
\item Often another method like the ``Average-Iron-Path''-method can
  be used to check results:  Is the derivation within an expected range?
\item Repeat your computation with a finer grid.  How does this
  influence your results?
\end{itemize}


\section{Parallelization}
\label{sec:parallelization}

This library targets at small to medium sized problems.\footnote{That
  means grids with up to about half a millon of nodes.} In this sphere
there is no great gain from parallelization.  Nevertheless it is
possible to exploit the power of multiple processors:
\begin{itemize}
\item Most of the time you won't analyse just a single design. What
  you want is to optimize your design.  So, use a parallel ant colony
  optimizer!
\item Although the library is limited to calculations in two
  dimensions it can be used for calculations in three dimension in
  some special cases.  This is done by cutting the device into slices
  and doing for every slice a two dimensional calculation.  These
  calculations can run in parallel.
\end{itemize}
Doing parallization on such a high level has a great advantage over a
parallized solver:  As there is only a very loose coupling at that
level it scales very well.

%%% Local Variables: 
%%% mode: latex
%%% TeX-master: "manual"
%%% ispell-local-dictionary: "en_US"
%%% End: 

%  LocalWords:  vertices

% -*- coding: utf-8 -*-
%
% Copyright © 2012-2016 Philipp Büttgenbach
%
% This file is part of ulphi, a CAE tool and library for
% computing electromagnetic fields.
%
% Permission is granted to copy, distribute and/or modify this
% document under the terms of the \gls{gnu} Free Documentation
% License, Version 1.3 or any later version published by the Free
% Software Foundation; with no Invariant Sections, no Front-Cover
% Texts, and no Back-Cover Texts.  A copy of the license is included
% in the section entitled GNU Free Documentation License.
%
% This manual is distributed in the hope that it will be useful, but
% WITHOUT ANY WARRANTY; without even the implied warranty of
% MERCHANTABILITY or FITNESS FOR A PARTICULAR PURPOSE.
%

\chapter{Examples}
\label{cha:examples}

\section{Coaxial Line}
\label{sec:coaxial-line}

The geometry for this example is shown in figure~\ref{fig:coaxial-line}.
It is a good example for validating the library because an analytical
solution is known.  \par The inner conductor carries the current~$I$
and the outer conductor carries the current~$-I$ -- both homogeneously
distributed over the conductor's surface.  The current density in the
outer conducter is
\begin{gather}
  J_3=-J_1\cdot\frac{r_1^2}{r_3^2-r_2^2}\text{~.}
\end{gather}
Using equation~\eqref{eq:3} the magnetic field intensity computes to
\begin{gather}
  H(r)= \frac{J_1}{2}
  \begin{cases}
    r & r \le r_1\text{~,} \\
    \frac{r_1^2}{r} & r_1 < r \le r_2\text{~,} \\
    \frac{r_1^2}{r}\cdot\left(1-\frac{r^2-r_2^2}{r_3^2-r_2^2}\right)
    & r_2 < r \le r_3\text{~.}
  \end{cases}
\end{gather}
From \eqref{eq:4} the magnetic vector potential is found to be
\begin{multline}
\label{eq:12}
  A(0)-A(r) = \frac{\mu_0J_1}{2} \\ \cdot
  \begin{cases}
    r^2/2 & r \le r_1\text{~,} \\
    r_1^2\left(\frac{1}{2}+\ln\frac{r}{r_1}\right)
    & r_1 < r \le r_2\text{~,} \\
    r_1^2\left(\frac{1}{2}+\ln\frac{r_2}{r_1}
      +(1+\frac{r_2^2}{r_3^2-r_2^2})\ln\frac{r}{r_2}
      +\frac{r^2-r_2^2}{2(r_3^2-r_2^2)}\right)
    & r_2 < r \le r_3\text{~.} \\
  \end{cases}
\end{multline}
Both solutions -- the numerical and \eqref{eq:12} -- are compared in
figure~\ref{fig:VectPotentialCoaxialLine} and agree very well.
\begin{figure}
  \begin{maxipage}
    \hfil
    \subfloat[Geometry]{
      \input{../images/Coaxial-Line.pdf_tex}
      \label{fig:coaxial-line}
    } \hfil
    \subfloat[Result]{
      \begin{tikzpicture}
        \begin{axis}[
          width=.6\linewidth,
          height=.3708203932499369\linewidth,
          xlabel=${r}/{\milli\metre}\quad\longrightarrow$,
          ylabel=$A\cdot\frac{\metre}{\micro\weber}\quad\longrightarrow$,
          grid=major,
          grid style={line width=.18mm},
          tick style={line width=.35mm},
          legend style={at={(0.95,0.95)},anchor=north east},
          line width=.35mm]
          \addplot[mark=diamond,only marks]
          table[x=x, y=NumA] {../datasets/coaxial_line.csv};
          \addplot[smooth,mark=none,line width=.7mm,style=dotted]
          table[x=x, y=SymA] {../datasets/coaxial_line.csv};
          \legend{numerical, analytical}
        \end{axis}
      \end{tikzpicture}
      \label{fig:VectPotentialCoaxialLine}
    }
    \caption{Coaxial line}
  \end{maxipage}
\end{figure}

\section{Simple iron core}
\label{sec:simple-iron-core}

In this example the coil is fed by a direct current.  After the
computation is done, the result is checked against the result gained
by the average-iron-path-method.  Also the field line plot
(figure~\ref{fig:SimpleIronCoreFieldLines}) looks reasonable.
\begin{figure}
  \centering
  \begin{tikzpicture}
    \begin{axis}[
      title={$A\cdot\frac{\metre}{\milli\weber}$},
      width=.95\linewidth,
      axis equal={true},
      xmin={0},
      xmax=50,
      ymin=-5,
      ymax=60,
      % xtick={-10,0,...,60},
      % ytick={-10,0,...,60},
      xlabel=$x/{\milli\metre}\quad\longrightarrow$,
      ylabel=$y/{\milli\metre}\quad\longrightarrow$,
      grid=major,
      grid style={line width=.18mm},
      tick style={line width=.35mm},
      line width=.35mm]
      %% iron core shape
      \addplot[gray,fill=gray!20,opacity=.33,line width=.7mm] 
      coordinates {
        (15,0) (40,0) (40,54) (0, 54) (0, 29) (15,29)} -- cycle;
      %% coils
      \addplot[copper,fill=copper,opacity=.33,line width=.7mm]
      coordinates {(5,0) (10,0) (10,24) (5,24)} -- cycle;
      \addplot[copper,fill=copper,opacity=.33,line width=.7mm]
      coordinates { (45,0) (50,0) (50,24) (45,24)} -- cycle;
      \addplot[contour gnuplot, line width=.7mm]
      table[x=x, y=y, z=potential] {../datasets/simple_iron_core.csv};
    \end{axis}
  \end{tikzpicture}
  \caption{Field lines inside the simple iron core}
  \label{fig:SimpleIronCoreFieldLines}
\end{figure}


\section{E-I-Transformer core}
\label{sec:inductor}

The device and the model setup for this example are shown in
figure~\ref{fig:ei-transformer-device} and
\ref{fig:ei-transformer-model}.
\begin{figure}
  \begin{maxipage}
    \hfil
    \subfloat[physical device]{
      \input{../images/Mantel-Transformator.pdf_tex}
      \label{fig:ei-transformer-device}
    } \hfil
    \subfloat[Model]{
      \input{../images/Mantel-Transformator-Modell.pdf_tex}
      \label{fig:ei-transformer-model}
    } \hfil
    \\[\bigskipamount]
    \hfil
    \subfloat[average core flux density]{
      \begin{tikzpicture}
        \begin{axis}[ width=.45\linewidth,
          height=.2781152949374527\linewidth,
          xlabel=$t/{\milli\second}\quad\longrightarrow$,
          ylabel=$B/{\tesla}\quad\longrightarrow$, grid=major, grid
          style={line width=.18mm}, tick style={line width=.35mm},
          legend style={at={(0.95,0.95)},anchor=north east}, line
          width=.35mm]
          \addplot[line width=.7mm] table[x=t,
          y=y1]{../datasets/E_I_transformer_core_time.csv}; 
          \addplot[smooth, line
          width=.7mm, style=dashed] table[x=t,
          y=y2]{../datasets/E_I_transformer_core_time.csv}; 
          \legend{waveform, fundamental}
        \end{axis}
      \end{tikzpicture}
      \label{fig:ei-transformer-result-time-domain}
    } \hfil \subfloat[spectrum]{
      \begin{tikzpicture}
        \begin{axis}[ width=.45\linewidth,
          height=.2781152949374527\linewidth, xlabel=harmonic
          order$\quad\longrightarrow$,
          ylabel=$B/{\tesla}\quad\longrightarrow$, grid=major, grid
          style={line width=.18mm}, tick style={line width=.35mm},
          line width=.35mm]
          \addplot[ycomb, line width=.7mm]
          table{../datasets/E_I_transformer_core_freq.csv};
        \end{axis}
      \end{tikzpicture}
      \label{fig:ei-transformer-result-harmonics}
    } \hfil
    \caption{E-I-transformer core}
    \label{fig:transformer}
  \end{maxipage}
\end{figure}
In this example the coil is fed by an alternating current.  As the
results show (figure~\ref{fig:ei-transformer-result-time-domain} and
\ref{fig:ei-transformer-result-harmonics}) the device is already
operated at the level of saturation.

\section{Wound Iron Core}
\label{sec:wound-iron-core}

This is the example which motivated the development of this library.
The iron core (figure~\ref{fig:wound-transformer-core}) has straight
and rounded sections.  This can be modeled using the different types
of grids.  \par In all examples before a current source
was used.  This is very uncommon in practice: Usually you are using a
voltage source.  Also you cannot specify a voltage source in ulphi,
you can get the same effect by using a inhomogeneous
\personname{Dirichlet} boundary condition to specify the total flux
passing throw the iron core.  After solving the task the magnetic
voltage is evaluated.
\begin{figure}
  \begin{maxipage}
    \hfil
    \subfloat[physical device]{
      \input{../images/Wound-Transformer-Core.pdf_tex}
    }
    \hfil
    \subfloat[model]{
      \input{../images/Example-Model.pdf_tex}
    }
    \hfil
    \\[\bigskipamount]
    \hfil
    \subfloat[magnetic voltage]{
      \begin{tikzpicture}
        \begin{axis}[ width=.45\linewidth,
          height=.2781152949374527\linewidth,
          xlabel=$t/{\milli\second}\quad\longrightarrow$,
          ylabel=$\gls{Um}/{\kilo\ampere}\quad\longrightarrow$, grid=major, grid
          style={line width=.18mm}, tick style={line width=.35mm},
          legend style={at={(0.95,0.95)},anchor=north east}, line
          width=.35mm]
          \addplot[line width=.7mm] table[x=t,
          y=y1]{../datasets/wound_iron_core_time.csv}; 
          \addplot[smooth, line
          width=.7mm, style=dashed] table[x=t,
          y=y2]{../datasets/wound_iron_core_time.csv}; \legend{waveform,
            fundamental}
        \end{axis}
      \end{tikzpicture}
      \label{fig:ei-transformer-result-time-domain}
    } \hfil \subfloat[spectrum]{
      \begin{tikzpicture}
        \begin{axis}[ width=.45\linewidth,
          height=.2781152949374527\linewidth, xlabel=harmonic
          order$\quad\longrightarrow$,
          ylabel=$\gls{Um}/{\kilo\ampere}\quad\longrightarrow$, grid=major, grid
          style={line width=.18mm}, tick style={line width=.35mm},
          line width=.35mm]
          \addplot[ycomb, line width=.7mm]
          table{../datasets/wound_iron_core_freq.csv};
        \end{axis}
      \end{tikzpicture}
      \label{fig:ei-transformer-result-harmonics}
    } \hfil
    \caption{Wound iron core}
    \label{fig:wound-transformer-core}
  \end{maxipage}
\end{figure}


%%% Local Variables: 
%%% mode: latex
%%% TeX-master: "manual"
%%% ispell-local-dictionary: "en_US"
%%% End: 


\appendix
fdl-1.3.tex

\printbibliography

\end{document}

%%% Local Variables: 
%%% mode: LaTeX
%%% ispell-local-dictionary: "en"
%%% End:
