% -*- coding: utf-8 -*-
%
% Copyright © 2012-2016 Philipp Büttgenbach
%
% This file is part of ulphi, a CAE tool and
% library for computing electromagnetic fields.
%
% Permission is granted to copy, distribute and/or modify this
% document under the terms of the \gls{gnu} Free Documentation
% License, Version 1.3 or any later version published by the Free
% Software Foundation; with no Invariant Sections, no Front-Cover
% Texts, and no Back-Cover Texts.  A copy of the license is included
% in the section entitled GNU Free Documentation License.
%
% This manual is distributed in the hope that it will be useful, but
% WITHOUT ANY WARRANTY; without even the implied warranty of
% MERCHANTABILITY or FITNESS FOR A PARTICULAR PURPOSE.
%

\chapter{Using Ulphi}
\label{cha:practical-aspects}

\section{Preprocessing}
\label{sec:preprocessing}

\subsection{Grids}
\label{sec:grids}

\begin{figure}
  \centering
  \begin{maxipage}
    \hfil
    \subfloat[cartesian]{
      \input{../images/Cart-Grid-Init.pdf_tex}}
    \hfil
    \subfloat[polar]{
      \input{../images/Polar-Grid-Init.pdf_tex}}
    \hfil
  \caption{Supported grid types}
  \label{fig:sup-grid-types}
  \end{maxipage}
\end{figure}
Currently, two types of grids are supported (figure~\ref{fig:sup-grid-types}):
\begin{itemize}
\item Cartesian grids
\item Polar grids
\end{itemize}
Some notes about grids:
\begin{itemize}
\item Grids are defined by their \gls{llc} and their \gls{urc}.
\item At creation time the global coordinate system becomes the grids
  local coordinate system.
\item Only rotating and shifting the grid make that global and local
  coordinate system differ.
\item Grids are default initialized by a material of no conductivity and the
  free space permeability.
\item Material definitions inside a grid may overlap. Valid is the
  last definition.
\end{itemize}

\subsection{Grid Boundaries}
\label{sec:grid-boundaries}

A grid's boundaries are named as shown in figure~\ref{fig:boundary-names}.
\begin{figure}
  \begin{maxipage}
    \hfil
    \begin{minipage}{.485\linewidth}
      \centering
      \input{../images/boundary-names.pdf_tex}
      \caption{A grid's boundary names}
      \label{fig:boundary-names}
    \end{minipage}
    \hfil
    \begin{minipage}{.485\linewidth}
      \centering
      \input{../images/edge-orientation.pdf_tex}
      \caption{Vertice numbering onto boundaries}
      \label{fig:Vertice-numbering-onto-boundaries}
    \end{minipage}
    \hfil
  \end{maxipage}
\end{figure}
Vertices onto the boundary have their own numbering as shown in
figure~\ref{fig:Vertice-numbering-onto-boundaries}.


\subsection{Grid Coupling}
\label{sec:grid-coupling}

When coupling boundaries some care must be taken by the user that it
is done correctly because the software can not detect all possible
error conditions:
\begin{itemize}
\item For a grid region at maximum two edges may be used as source
  edges in couplings.
\item These links must be defined on parallel edges.
\end{itemize}
After linking two edges the source edge will be hidden by the target
edge.  Figure~\ref{fig:GridCouplings} shows some examples.
\begin{figure}
  \begin{maxipage}
    \centering
    \hfil
    \subfloat[Correct forwarding for the corner vertex]{
      \input{../images/edge-linking-1.pdf_tex}} \hfil
    \subfloat[Correct forwarding for the corner vertex]{
      \input{../images/edge-linking-2.pdf_tex}} \hfil \\[2\bigskipamount]
    \subfloat[Wrong! The corner vertex exists twice]{
      \input{../images/edge-linking-3.pdf_tex}}
    \caption{Examples for grid couplings}
    \label{fig:GridCouplings}
  \end{maxipage}
\end{figure}
Cyclic coupling of four grids is currently supported and correctly
handled when it is detected.  This support may be removed in the
future because it is a huge complication. So, avoid cyclic coupling of
four grids!


\subsection{Vertex Distributions}

The distribution of vertices inside a grid is defined by vertex
distributions.  Two types are currently implemented:
\begin{itemize}
\item A linear vertex distribution and
\item an exponential vertex distribution.
\end{itemize}
If you have special demands you can implement your own vertex
distribution type by deriving from the \texttt{VertexDistribution}
base class.


\subsection{Shapes}

Shapes are used to define materials and current carrying regions
inside a grid.  The following shapes are implemented:
\begin{description}
\item[Rectangle] A rectangle is a shape with four edges parallel to
  the coordinate system's axes.  This implies that in a polar system a
  rectangle has two rounded edges.
\item[Circle] is defined by its center point and its radius.
\end{description}
If you have special demands you can implement your own shape types by
deriving from the \texttt{Shape} base class.

\subsection{Reluctivity Models}
\label{sec:material-models}

Several reluctivity models are supported which are described in the
following.
\begin{description}
\item[Constant reluctivity] The constant reluctivity model is simply a
  constant value. An example for an material with a constant
  reluctivity is air.
\item[Arctan model] This model has its origin in the FLUX® \gls{cad}
  package \parencite{Cedrat:2006}. It uses an arctan function to model the
  nonlinear behaviour of materials like iron.
  \begin{gather}
    \label{eq:39}
    B(H)=\mu_0H+\frac{2B_{\mathrm{i,s}}}{{\pi}}\arctan\left(
    \frac{{\pi}(\mu_{\mathrm{r}}-1)\mu_0H}{2B_{\mathrm{i,s}}}\right)
  \end{gather}
  with
  \begin{align*}
    \mu_{\mathrm{r}}&\quad\text{rel. permeability at the origin} \\
    B_{\mathrm{i,s}}&\quad\text{saturation polarization}
  \end{align*}
  This model may be initialized from these two parameters or from
  measured data.  In the last case the
  \personname{Levenberg}-\personname{Marquardt} procedure is used to
  find a set of good parameters.  \par Although using such a model
  might look like a good idea on a first glance because a strictly
  monotone curve helps the \personname{Newton} process to converge, this model
  has some drawbacks:
  \begin{itemize}
  \item The $B$-$H$-characteristic is modeled but for solving equation
    \eqref{eq:41} the $\nu$-$B$-characteristic is required.  Because
    there is no analytical expression for the
    $\nu$-$B$-char\-ac\-ter\-is\-tic resulting from this model,
    \personname{Newton}'s procedure must be used.
  \item The calculation of the derivative $\partial\nu/\partial B$
    yields a lengthy expression which is hard to check for
    correctness.
  \item Most of the time real world magnetization curves are not
    fitted that well by this model.
  \end{itemize}
\item[Square root model] This model uses a square root function to
  approximate the $B$-$H$-curve:
  \begin{align}
    \label{eq:42}
    B(H) &=
    \mu_0H+B_{\mathrm{i,s}}\frac{H_a+1-\sqrt{(H_a+1)^2-4H_a(1-a)}}{2(1-a)}
    \intertext{with}
    H_a&=\mu_0H\frac{\mu_r-1}{B_{\mathrm{i,s}}}
  \end{align}
  In all other aspects this model is similar to the previous model.
\item[Spline model] This model uses a spline to interpolate measured
  values.  If extrapolation is required, a tangent line through the
  first or last data point is used.  This model is recommended for
  everyday use.
\end{description}
Figure~\ref{fig:reluct-model-perf} shows how these models perform with
real world data.
\begin{figure}
  \begin{maxipage}
    \centering
    \subfloat[Arctan model]{
      \begin{tikzpicture}
        \begin{semilogyaxis}[
          height=.275\textheight,
          width=.44495934690622113\textheight,
          xlabel=${B}/{\tesla}\quad\longrightarrow$,
          ylabel=$\nu\cdot\frac{\henry}{\metre}\quad\longrightarrow$,
          grid=major,
          grid style={line width=.18mm},
          tick style={line width=.35mm},
          legend style={at={(0.04,0.96)},anchor=north west},
          line width=.35mm]
          \addplot[mark=diamond,only marks]
          table[x=B, y=NuOrig] {../datasets/M400_50A-arctan.csv};
          \addplot[smooth,mark=none,line width=.7mm,style=dotted]
          table[x=B, y=NuModel] {../datasets/M400_50A-arctan.csv};
          \legend{orig. data, arctan model}
        \end{semilogyaxis}
      \end{tikzpicture}
    }\\[\fill]
    \subfloat[Square root model]{
      \begin{tikzpicture}
        \begin{semilogyaxis}[
          height=.275\textheight,
          width=.44495934690622113\textheight,
          xlabel=${B}/{\tesla}\quad\longrightarrow$,
          ylabel=$\nu\cdot\frac{\henry}{\metre}\quad\longrightarrow$,
          grid=major,
          grid style={line width=.18mm},
          tick style={line width=.35mm},
          legend style={at={(0.04,0.96)},anchor=north west},
          line width=.35mm]
          \addplot[mark=diamond,only marks]
          table[x=B, y=NuOrig] {../datasets/M400_50A-sqrt.csv};
          \addplot[smooth,mark=none,line width=.7mm,style=dotted]
          table[x=B, y=NuModel] {../datasets/M400_50A-sqrt.csv};
          \legend{orig. data, sqrt model}
        \end{semilogyaxis}
      \end{tikzpicture}
    }\\[\fill]
    \subfloat[Spline model]{
      \begin{tikzpicture}
        \begin{semilogyaxis}[
          height=.275\textheight,
          width=.44495934690622113\textheight,
          xlabel=${B}/{\tesla}\quad\longrightarrow$,
          ylabel=$\nu\cdot\frac{\henry}{\metre}\quad\longrightarrow$,
          grid=major,
          grid style={line width=.18mm},
          tick style={line width=.35mm},
          legend style={at={(0.04,0.96)},anchor=north west},
          line width=.35mm]
          \addplot[mark=diamond,only marks]
          table[x=B, y=NuOrig] {../datasets/M400_50A-spline.csv};
          \addplot[smooth,mark=none,line width=.7mm,style=dotted]
          table[x=B, y=NuModel] {../datasets/M400_50A-spline.csv};
          \legend{orig. data, spline model}
        \end{semilogyaxis}
      \end{tikzpicture}
    }
    \caption[Reluctivity models and how they perform with real world
    data]{
      Reluctivity models and how they perform with real world
      data. The material is M400-50A.}
    \label{fig:reluct-model-perf}
  \end{maxipage}
\end{figure}


\subsection{Material and Load Region Boundaries Inside a Grid}

Inside a grid the boundaries of material and load regions are
approximated in order to match the grid
(figure~\ref{fig:ApproxShapes}).
\begin{figure}
  \begin{maxipage}
    \centering \subfloat[real geometry]{
      \input{../images/Approx-Shapes-1.pdf_tex} } \hfil \subfloat[real
    geometry]{ \input{../images/Approx-Shapes-2.pdf_tex} } \hfil
    \subfloat[approximated geometry]{
      \input{../images/Approx-Shapes-3.pdf_tex} }
    \caption[Approximation of regions inside a grid]{ Approximation of
      regions inside a grid. Solid: primary grid; dashed: secondary
      grid.}
    \label{fig:ApproxShapes}
  \end{maxipage}
\end{figure}
Errors due to this approximation can be kept small by using a
reasonable fine grid.  If you want to suppress this approximation, you
should carefully line up your geometry with the secondary grid.


\section{Solving}
\label{sec:solving}

There are three different solving procedures implemented:
\begin{description}
\item[Linear solver] This solver is used if all material properties
  are field independent.
\item[Iterative solver] This is a simple iterative solver.  Good luck!
\item[\personname{Newton} iteration] This solver uses the
  sophisticated \personname{Newton} procedure and is the best choice
  for nonlinear tasks (field dependent relucitivty).
\end{description}

\section{Post processing}
\label{sec:postprocessing}

From the grid graph the vertice's positions and potentials can be
extracted.
In the python interface this is done by:
\begin{lstlisting}[language=python]
# ... initialize grid ... \pythonmarginlabel{}
grid_compiler = GridCompiler(grid)
grid_graph = grid_compiler()
grid_graph.Newton(1e-5, 24)

# Access results
grid_graph.positions
grid_graph.potentials
\end{lstlisting}

\subsection{Using a Path}
\label{sec:using-path}

\begin{figure}
  \centering
  \begin{maxipage}
     \hfil
    \subfloat[evaluate flux]{
      \input{../images/calc-flux-with-path.pdf_tex}}
    \hfil
    \subfloat[evaluate magnetic voltage]{
      \input{../images/calc-magnetic-voltage-with-path.pdf_tex}}
    \hfil
  \caption{Using a path for postprocessing}
  \label{fig:UsePath}
  \end{maxipage}
\end{figure}
To evaluate the total flux\footnote{The total flux can be computed
from the average flux density.} and the magnetic voltage, a path is
used.
\begin{lstlisting}[language=python,texcl]
# Initialize a path \pythonmarginlabel{}
path = PathInGridGraph(grid_graph, pointlist)

# Access average flux density
path.average_flux_density

# Access the magnetic voltage
path.magnetic_voltage
\end{lstlisting}

\subsection{Check Your Results}

It is always important to check the results:
\begin{itemize}
\item Does the field line plot look reasonable?
\item Often another method like the ``Average-Iron-Path''-method can
  be used to check results:  Is the derivation within an expected range?
\item Repeat your computation with a finer grid.  How does this
  influence your results?
\end{itemize}


\section{Parallelization}
\label{sec:parallelization}

This library targets at small to medium sized problems.\footnote{That
  means grids with up to about half a millon of nodes.} In this sphere
there is no great gain from parallelization.  Nevertheless it is
possible to exploit the power of multiple processors:
\begin{itemize}
\item Most of the time you won't analyse just a single design. What
  you want is to optimize your design.  So, use a parallel ant colony
  optimizer!
\item Although the library is limited to calculations in two
  dimensions it can be used for calculations in three dimension in
  some special cases.  This is done by cutting the device into slices
  and doing for every slice a two dimensional calculation.  These
  calculations can run in parallel.
\end{itemize}
Doing parallization on such a high level has a great advantage over a
parallized solver:  As there is only a very loose coupling at that
level it scales very well.

%%% Local Variables: 
%%% mode: latex
%%% TeX-master: "manual"
%%% ispell-local-dictionary: "en_US"
%%% End: 

%  LocalWords:  vertices
