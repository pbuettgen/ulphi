% -*- coding: utf-8 -*-
%
% Copyright © 2012-2016 Philipp Büttgenbach
%
% This file is part of ulphi, a CAE tool and library for
% computing electromagnetic fields.
%
% Permission is granted to copy, distribute and/or modify this
% document under the terms of the \gls{gnu} Free Documentation
% License, Version 1.3 or any later version published by the Free
% Software Foundation; with no Invariant Sections, no Front-Cover
% Texts, and no Back-Cover Texts.  A copy of the license is included
% in the section entitled GNU Free Documentation License.
%
% This manual is distributed in the hope that it will be useful, but
% WITHOUT ANY WARRANTY; without even the implied warranty of
% MERCHANTABILITY or FITNESS FOR A PARTICULAR PURPOSE.
%

\chapter{Discretization -- The  \glsentrytext{fit}}
\label{cha:discretization}

Unfortunately, the equations from the previous section can only be
solved in a limited number of cases in a closed form.  In the general
case they are discretized, that is the computation domain is
subdivided by a grid or mesh, and solved numerically. Here we are
going to use the \gls{fit}. \par
Originally the \gls{fit} was developed to compute especially high
frequency electromagnetic fields as they accure in radio frequency
particle accelerators \parencite{Weiland:1977}.  In that case it uses
the full set of equations~\eqref{eq:1} throw \eqref{eq:2}.
Nevertheless it can be seen as a much more general method and adapted
to other field equations as it is shortly pointed out by \cite[section
3.4.2]{vanRienen:2010}. \par
Compared to the \gls{fdm} it has some important advantages:
\begin{itemize}
\item The resulting system matrix is always symmetric, also for non
  equidistant grids.  This is an important property on which many
  sparse system solvers rely.
\item Coupling grids of different type (cartesian and polar) is very
  easy and natural to implement and there is a smooth transition in
  the coefficients.
\end{itemize}
\par
The finite integral method starts with the equations from section 2 in
their integral form. The computation domain is subdivided by a grid
and the grid’s edges become the integral’s integration path
(figure~\ref{fig:FIT-grid-3D}).
\begin{figure}
  \begin{maxipage}
    \centering
    \subfloat[Three dimensional \gls{fit} grid for equation~\eqref{eq:4}] {
      \input{../images/FIT-grid-3D.pdf_tex}
      \label{fig:FIT-grid-3D}
    }
    \\[\bigskipamount]
    \subfloat[Two dimensional projection to the $x$-$y$-plane] {
      \input{../images/FIT-grid-2D.pdf_tex}
      \label{fig:FIT-grid-2D}
    }
    \caption{Elementary \gls{fit} grid in magnetostatic applications}
  \end{maxipage}
\end{figure}
In the case where a computation in two dimensions is sufficient, this
grid is projected to a plane (figure~\ref{fig:FIT-grid-2D}).
All entities in this grid are referred to by their row and column index:
\begin{description}
\item[$m$] row index $m\in\left[0\dots\xmax{m}\right]$
\item[$n$] column index $n\in\left[0\dots\xmax{n}\right]$
\end{description}
Using equation~\eqref{eq:4} in its integral form the flux density
inside such a grid computes as
\begin{gather}
  \label{eq:9}
  \left[-A_z(m,n)+A_z(m+1,n)\right]\cdot\gls{lD}=\Phi_y(m,n)\text{~,}
  \intertext{and with}
  \Phi_y(m,n)=B_x(m,n)\cdot\gls{lD}\cdot\Delta_y(m)
\end{gather}
it follows
\begin{subequations}
  \begin{align}
    B_x(m,n)&=\frac{1}{\Delta_y(m)}\left[A_z(m+1,n)-A_z(m,n)\right]
    \intertext{Analogously it is}
    \label{eq:10}
    B_y(m,n)&=\frac{1}{\Delta_x(n)}\left[A_z(m,n)-A_z(m,n+1)\right]\text{~.}
  \end{align}
\end{subequations}
\par The outer rotation in equation~\eqref{eq:5} forms a
secondary or inner grid.  This grid has the following properties
\footnote{All the secondary grid's properties are marked by $\tilde{{}}$.}:
\begin{align}
  \tilde{\Delta}_{x}(n)
  &=\frac{1}{2}\cdot\left[x(n+1)-x(n-1)\right] \\
  \tilde{\Delta}_{y}(m)
  &=\frac{1}{2}\cdot\left[y(m+1)-y(m-1)\right]
  \intertext{A grid cell's area is given by}
  \label{eq:13}
  \tilde{S}(m,n)
  &=\tilde{\Delta}_x(n)\cdot\tilde{\Delta}_y(m)
\end{align}
Polar coordinates:
\begin{align}
  \label{eq:14}
  \tilde{\Delta}_{r}(n)&=\frac{1}{2}\left[r(n+1)-r(n-1)\right] \\
  \tilde{\Delta}_{\phi}(m)&=\frac{1}{2}\left[\phi(m+1)-\phi(m-1)\right]\\
  \tilde{r}(n) &= \frac{1}{2}\left[r(n+1)+r(n)\right] \\
  \tilde{S}(m,n) &= \frac{\tilde{\Delta}_{\phi}(m)}{2}
  \left[\tilde{r}^{2}(n+1)-\tilde{r}^{2}(n)\right]
\end{align}
Writing out the equations for the grid in figure~\ref{fig:FIT-grid-2D}
yields:
\begin{align*}
  B_x(\tilde{0})&=\frac{1}{\Delta_y(1)}\left[A_z(4)-A_z(1)\right] \\
  B_x(\tilde{3})&=\frac{1}{\Delta_y(4)}\left[A_z(7)-A_z(4)\right] \\
  B_y(\tilde{0})&=\frac{1}{\Delta_x(3)}\left[A_z(3)-A_z(4)\right] \\
  B_y(\tilde{1})&=\frac{1}{\Delta_x(4)}\left[A_z(4)-A_z(5)\right]
\end{align*}
And finally:
\begin{multline*}
  B_x(\tilde{0})\nu_x(\tilde{0})\tilde{\Delta}_x(\tilde{0})
  +B_y(\tilde{1})\nu_y(\tilde{1})\tilde{\Delta}_y(\tilde{1}) \\
  -B_x(\tilde{3})\nu_x(\tilde{3})\tilde{\Delta}_x(\tilde{3})
  -B_y(\tilde{0})\nu_y(\tilde{0})\tilde{\Delta}_y(\tilde{0}) = J(4)\cdot\tilde{S}(4)
\end{multline*}
Now the following coefficients are introduced:
\begin{align}
  \label{eq:16}
  g_y(1)&=\frac{\tilde{\Delta}_x(\tilde{0})}{\Delta_y(1)}\nu_x(1) &
  g_x(4)&=\frac{\tilde{\Delta}_y(\tilde{1})}{\Delta_x(4)}\nu_y(4) \\
  g_y(4)&=\frac{\tilde{\Delta}_x(\tilde{3})}{\Delta_y(4)}\nu_x(4) &
  g_x(3)&=\frac{\tilde{\Delta}_y(\tilde{0})}{\Delta_x(3)}\nu_y(3)
\end{align}
And now:
\begin{multline}
  \label{eq:17}
  \left[g_y(1)+g_x(4)+g_y(4)+g_x(3)\right] A_z(4) \\
  -g_y(1)A_z(1)-g_x(4)A_z(5)-g_y(4)A_z(7)-g_xA_z(3) = J(4)\tilde{S}(4)
\end{multline}
This equation can be expressed in matrix form as
\begin{gather}
  \label{eq:18}
  \matr{G}\vect{A}=\vect{J}\circ\vect{S}=\vect{\Theta}
\end{gather}
what is especially useful for larger grids.  The system matrix has a
special structure which makes it easy to directly generate it from the
grid (algorithm~\ref{alg:1}).\footnote{If you are familiar to the
  nodal analysis of electrical networks, you will probably notice that
  the rules are quite similar.}
\begin{algorithm}
  \ForAll{vertices $v$}{
    \ForAll{out edges $e$ of $v$}{
      $\tilde{e}=\text{secondary grid's crossing edge}$\;
      $l_{\text{sec}}=\operatorname{length}(\tilde{e})$\;
      $l_{\text{prim}}=\operatorname{length}(e)$\;
      $g=\nu\cdot l_{\text{sec}}/l_{\text{prim}}$\;
      %
      $\vect{G}(\operatorname{source}(e),\operatorname{target}(e))=-g$\;
      %
      $\vect{G}(\operatorname{source}(e),\operatorname{source}(e))+=g$\;
    }
  }
\caption{Generation of matrix entries for a grid's inner node.}
\label{alg:1}
\end{algorithm}


\section{Field Dependent Reluctivity}
\label{sec:field-depend-reluct}

Nonlinear material properties appear frequently in magnetic
calculations.  So it is nessesary to deal with them.  Usually an
iterative process is used to solve the nonlinear equations.

\subsection{Simple Iteration}
\label{sec:simple-iteration}

The simplest way to handle nonlinear material properties is to
iteratively resolve \eqref{eq:18} and adjust the reluctivity after each
step.  It is also possible to implement a damped variant where a
weighted average of the solution and previous solution is computed.
This algorithm has the following advantages and
disadvantages:\footnote{See \parencite[chapter 5]{Silvester:1996} for
  a more detailed discussion}
\begin{description}
\item[Advantage]
  \begin{itemize}
  \item Simple, no derivation required
  \end{itemize}
\item[Disadvantage]
  \begin{itemize}
  \item slow convergence
  \item no certain way for error estimation
  \end{itemize}
\end{description}


\subsection{\personname{Newton} Iteration}
\label{sec:newton-iteration}

The \personname{Newton} procedure \parencite{Wikipedia:Newton} is
described by the following equation: \index{Newton procedure|emph}
\begin{gather}
  \label{eq:21}
  \vect{A}(q+1)=\vect{A}(q)-\jacmatr^{-1}\cdot\vect{f}(\vect{A}(q))\text{~.}
\end{gather}
with
\begin{gather*}
  \vect{f}(\vect{A}(q))=\matr{G}\vect{A}(q)-\vect{\Theta}
\end{gather*}
and
\begin{align*}
  \jacmatr 
  & \quad\text{\personname{Jacobi} matrix of $\vect{f}(\vect{A})$,} \\
  \gls{q} & \quad\text{the \glsdesc{q}.}
\end{align*}
Of course we want to avoid computing the inverse of the
\personname{Jacobi} matrix \jacmatr{} -- a big sparse matrix -- due to its
computational costs.  So we reformulate \eqref{eq:21} as
\begin{gather}
  \label{eq:41}
  [\upDelta\vect{A}](q)=\vect{A}(q+1)-\vect{A}(q)\text{~,} \\
  \jacmatr\cdot[\upDelta\vect{A}](q)=-\vect{f}(\vect{A}(q))\text{~.}
\end{gather}
This system of equations can be solved for $[\upDelta\vect{A}](q)$ by some
suitable method like the conjugate gradient method.
\par To solve this system of equations we need the Jacobian matrix's
entries.  The \personname{Jacobi} matrix
\index{Jacobi matrix@\personname{Jacobi} matrix|emph} is defined as
\begin{gather}
  \label{eq:29}
  \jacmatr =
  \begin{pmatrix}
    \partdiff{f_0}{A_0} & \partdiff{f_0}{A_1} & \dots &
    \partdiff{f_0}{A_{\xmax{n}}} \\
    \partdiff{f_1}{A_0} & \partdiff{f_1}{A_1} & \dots &
    \partdiff{f_1}{A_{\xmax{n}}} \\
    \vdots & \vdots & \ddots & \vdots \\
    \partdiff{f_{\xmax{m}}}{A_0} & \partdiff{f_{\xmax{m}}}{A_1} &
    \dots & \partdiff{f_{\xmax{m}}}{A_{\xmax{n}}}
  \end{pmatrix}\text{~.}
\end{gather}
A single entry of this matrix is
\begin{gather}
  \label{eq:23}
  f_m=-\Theta_m+\sum_{k=0}^{\xmax{n}}g_{mk}A_k
\end{gather}
with the derivation
\begin{gather}
  \label{eq:24}
  \partdiff{f_m}{A_n}
  =\left(\sum_{k=0}^{\xmax{n}}\partdiff{}{A_n}g_{mk}A_k\right)
\end{gather}
Let's analyse this derivation in parts.
\begin{gather}
  \label{eq:26}
  \sum_{k=0}^{\xmax{n}}\partdiff{}{A_n}g_{mk}A_k
  =\sum_{k=0}^{\xmax{n}}\left(\partdiff{g_{mk}}{A_n}A_k
    +g_{mk}\partdiff{A_k}{A_n}\right)
\end{gather}
The second term evaluates to
\begin{gather}
  \label{eq:27}
  g_{mk}\partdiff{A_k}{A_n}=
  \begin{cases}
    g_{mk} & k = n \\
    0     & k \ne n
  \end{cases}
\end{gather}
and thereby only appears on the diagonal.  In the first term it is
nessesary to apply the chain rule of differentiation because $g_{mk}$
depends indirectly on $A_n$:
\begin{gather}
  \label{eq:28}
  \partdiff{g_{mk}}{A_n}A_k=\partdiff{g_{mk}}{B}\partdiff{B}{A_n}\cdot
  A_k=\pm\frac{\tilde{\Delta}}{\Delta^2}\partdiff{\nu}{B}\cdot A_k
\end{gather}
Despite this lengthy calculation the final result is rather simple.
First of all, the \personname{Jacobi} matrix has the same structure as
the system matrix \matr{G}:  Elements which are zero in \matr{G} are also zero
in \jacmatr.
Further it turns out that every edge -- beside the already introduced
coefficient $g$ -- has attached a jacobi coefficient $g_{\mathrm{J}}$ which is
\begin{gather}
  \label{eq:11}
  g_{\mathrm{J}}=g+\frac{\tilde{\Delta}}{\Delta}B\partdiff{\nu}{B}
\end{gather}
With this definition the \personname{Jacobi} matrix can be generated
in the same way as the system matrix (algorithm~\ref{alg:1}).

\subsubsection{Damped \personname{Newton} Procedure}
\label{sec:damp-newt-proc}

It is possible to introduce a \glsdesc{delta}~\gls{delta} into the
\personname{Newton} process:
\begin{gather}
  \label{eq:30}
  \vect{A}(q+1)=\vect{A}(q)+\delta\cdot[\upDelta\vect{A}](q)
\end{gather}
The damping factor is chosen so that
\begin{gather}
  \label{eq:31}
  \delta\le\min\left(1,
    \frac{\vectnorm{[\upDelta\vect{A}](q-1)}}{\vectnorm{[\upDelta\vect{A}](q)}}\right)\text{~.}
\end{gather}
So, if the error becomes bigger in the current step than in the last
step then the damping factor is reduced.

\section{Boundary Conditions}
\label{sec:bound-treatm-cond}

\subsection{\personname{Neumann} Boundary Condition}
\label{sec:neumann}

The \personname{Neumann} condition is about setting the first
derivative to a prescribed value.  In the case considered here, this
means nothing else then presetting the magnetic flux density parallel
to the boundary.
\begin{figure}
  \centering
  \input{../images/Neumann-Boundary.pdf_tex}
  \caption{Boundary with a \personname{Neumann} condition}
  \label{fig:neumann-boundary}
\end{figure}
Figure~\ref{fig:neumann-boundary} shows how the \personname{Neumann}
condition is applied in the context of finite integrals.  The
integration path around the boundary vertex is in parts a ``ghost'' path
because it lies outside the computation domain.  It is choosen in such
a way that the surounded vertex is at its center.  \par The relevant
equations for a \personname{Neumann} vertex are:
\begin{align*}
  B_x(3)&=\frac{1}{\Delta_y(3)}\left[A_z(6)-A_z(3)\right]  \\
  B_y(3)&=\frac{1}{\Delta_x(3)}\left[A_z(3)-A_z(4)\right]  \\
  B_x(0)&=\frac{1}{\Delta_y(0)}\left[A_z(3)-A_z(0)\right]
\end{align*}
And finally:
\begin{multline}
  \label{eq:19}
  B_y(3)g_x(3)-B_x(3)g_y(3)+B_x(0)g_y(0) \\
  =J_z(3)\tilde{S}(3)+B_{\text{out}}\nu_{\text{out}}l_{\text{out}}
\end{multline}
So, the \personname{Neumann} boundary condition modifies the
equation's right hand side.  In the case of the homogenious
\personname{Neumann} boundary condition
($B_{\text{out}}=\unit{0}{\tesla}$) it is the same equation as for a
grid's inner vertex.  That means the homogenious \personname{Neumann}
condition arises automatically if no other condition is set.
\emph{Important:}  The sign of the right hand side's modification term
depends onto the boundary:
\begin{gather*}
  \text{sign}=
  \begin{cases}
    -1 & \text{the boundary is parallel to the axis} \\
    +1 & \text{the boundary is \emph{anti}parallel to the axis}
  \end{cases}
\end{gather*}

\subsection{\personname{Dirichlet} Boundary Condition}
\label{sec:dirichlet}

The \personname{Dirichlet} boundary condition describes the case that no flux
crosses the boundary.  In other words the flux perpendicular to the
boundary is zero or -- expessed in terms of the magnetic vector
potential -- the magnetic vector potential along the boundary has a
constant prescribed value (figure~\ref{fig:diri-condition}).
\begin{figure}
  \centering
  \input{../images/Dirichlet-Boundary.pdf_tex}
  \caption{Boundary with a \personname{Dirichlet} condition}
  \label{fig:diri-condition}
\end{figure}
 Because the potential is known in advance
the vertice's equation can be eleminated from \eqref{eq:18}:
\begin{multline}
  \left[g_x(n)+g_y(n)+g_x(n-1)+g_y(n-\xmax{m})\right]A_z(n) \\
  -g_x(n)A_z(n+1)-g_y(n)A_z(n+\xmax{m})-g_y(n-\xmax{m})A_z(n-\xmax{m}) \\
  = J_n\tilde{S}_n+g_x(n-1)A_z(n-1)
\end{multline}
This leads to algorithm~\ref{alg:2} for adding \personname{Dirichlet}
vertices to the equation system.
\begin{algorithm}
  \ForAll{$\operatorname{out\_edges}(\text{\personname{Dirichlet}
      vertex})$}{
    Add $g\cdot A_z$ to the equation's right hand side of the target
    vertex\;
  }
  \caption{Add \personname{Dirichlet} vertices to the equation system}
  \label{alg:2}
\end{algorithm}

\section{Coupled Boundaries}
\label{sec:coupled-boundaries}

With finite integrals the implementations of boundary coupling is so
natural that it is almost not worth to mention it at all.
\begin{figure}
  \centering
  \input{../images/Interface-polar-cartesian.pdf_tex}
  \caption{Interface between a cartesian and polar grid patch}
  \label{fig:polar-cartesian}
\end{figure}
Figure~\ref{fig:polar-cartesian} shows how two grids patches can be
coupled.  Only the calculation of the integration path'es length and
the surrounded surface area must be adjusted.  
% \par Coming soon: Antiperiodic coupling.


%%% Local Variables: 
%%% mode: latex
%%% TeX-master: "manual"
%%% ispell-local-dictionary: "en_US"
%%% End: 
