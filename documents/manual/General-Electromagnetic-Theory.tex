% -*- coding: utf-8 -*-
%
% Copyright © 2012-2016 Philipp Büttgenbach
%
% This file is part of ulphi, a CAE
% tool and library for computing electromagnetic fields.
%
% Permission is granted to copy, distribute and/or modify this
% document under the terms of the \gls{gnu} Free Documentation
% License, Version 1.3 or any later version published by the Free
% Software Foundation; with no Invariant Sections, no Front-Cover
% Texts, and no Back-Cover Texts.  A copy of the license is included
% in the section entitled GNU Free Documentation License.
%
% This manual is distributed in the hope that it will be useful, but
% WITHOUT ANY WARRANTY; without even the implied warranty of
% MERCHANTABILITY or FITNESS FOR A PARTICULAR PURPOSE.
%

\chapter{General Electromagnetic Theory}
\label{cha:gener-electr-theory}

Electromagnetic fields are generally described by
\personname{Maxwell}'s equations.  Using \personname{Gauss}' and
\personname{Stockes}' theorem, these equations can be converted from
their differential form into an integral form:
\begin{align}
  \label{eq:1}
  \nabla\cdot\vect{\gls{B}}&=0 &\Leftrightarrow&&
  \oiinteg{\gls{S}}{\vect{B}}{\vect{S}}&=0 \\
  %% 
  \nabla\cdot\vect{D}&=\varrho &\Leftrightarrow&&
  \oiinteg{\gls{S}}{\vect{\gls{D}}}{\vect{S}}&=\gls{Q} \\
  %% 
  \label{eq:7}
  \nabla\vectimes\vect{\gls{E}}&=-\partdiff{\vect{B}}{t} &\Leftrightarrow&&
  \ointeg{l}{\vect{\gls{E}}}{\vect{l}}&=-\partdiff{\gls{Phi}}{t} \\
  %%
  \label{eq:2}
  \nabla\vectimes\vect{\gls{H}}&=\vect{\gls{J}}+\partdiff{\vect{D}}{t}
  &\Leftrightarrow&&
  \ointeg{l}{\vect{H}}{\vect{l}}&= \gls{Theta} + \partdiff{\gls{Psi}}{t}
\end{align}
In low frequency applications, that means a frequency below about one
kilohertz, displacement currents can be generally ignored.  There\-by
equation~\eqref{eq:2} simplyfies to
\begin{gather}
  \label{eq:3}
  \nabla\vectimes\vect{H}=\vect{J}\text{~.}
\end{gather}
Next a magnetic vector potential is introduced:
\begin{gather}
  \label{eq:4}
  \nabla\vectimes\gls{A}=\vect{B}\quad\Leftrightarrow\quad
  \ointeg{l}{\vect{A}}{\vect{l}}=\Phi\text{~.}
\end{gather}
By this definition equation~\eqref{eq:1} is fulfilled automatically.
This potential is ambiguous because
\begin{gather}
  \label{eq:8}
  \nabla\vectimes\left(\vect{A}+\nabla F\right)
  =\nabla\vectimes\vect{A}+\nabla\vectimes\left(\nabla F\right)
  =\nabla\vectimes\vect{A}\text{~,}
\end{gather}
where $F$ is some scalar field.
To make this potential unique
\begin{itemize}
\item a gauge like $\nabla\cdot\vect{A}=0$ can be enforced
\item the potential may be preset at least at one point inside the
  computation domain or its boundary.  This is the case when a
  \personname{Dirichlet} boundary condition
  (section~\ref{sec:dirichlet}) is defined.
\end{itemize}
Using this potential in equation~\eqref{eq:3} yields
\begin{gather}
  \label{eq:5}
  \nabla\vectimes\left(\nu\nabla\vectimes\vect{A}\right)=\vect{J}\text{~.}
\end{gather}
\par Eddy currents are not regarded in the following in order to keep
things simple.  For many devices this is a valid procedure:
\begin{itemize}
\item Iron cores of transformers (and generally of electrical
  machines) are subdevided into single sheets to suppress eddy currents.
\item Windings are made of thin wire in order to suppress eddy currents.
\end{itemize}

%%% Local Variables: 
%%% mode: latex
%%% TeX-master: "manual"
%%% ispell-local-dictionary: "en"
%%% End: 
