% -*- coding: utf-8 -*-
%
% Copyright © 2012-2016 Philipp Büttgenbach
%
% This file is
% part of ulphi, a CAE tool and library for computing electromagnetic
% fields.
%
% Permission is granted to copy, distribute and/or modify this
% document under the terms of the \gls{gnu} Free Documentation
% License, Version 1.3 or any later version published by the Free
% Software Foundation; with no Invariant Sections, no Front-Cover
% Texts, and no Back-Cover Texts.  A copy of the license is included
% in the section entitled GNU Free Documentation License.
%
% This manual is distributed in the hope that it will be useful, but
% WITHOUT ANY WARRANTY; without even the implied warranty of
% MERCHANTABILITY or FITNESS FOR A PARTICULAR PURPOSE.
%

\chapter[Important techniques in electromagnetic modelling]{
Important techniques in electromagnetic modelling of electrical
machines}

\label{cha:import-techn-electr}


\section{Modelling blocks of laminated material}
\label{cha:magnetic-resistance}
\index{Laminated material!modelling|emph}

When modelling blocks of laminated material, it is very impractical to
impossible to model each layer separately: Standard thicknesses for
electrical steel are in the range of 0.2~millimetre to
1~millimetre but in some applications even thinner sheets
are used.  Therefore laminated materials are modelled as a homogeneous
material which mimics the behaviour of the laminated material.  In the
following the properties for such an homogenized material are derived
using a magnetic circuit model of the laminated material.\footnote{This topic
is also treated in i. e. \cite[][Section 7.4]{Salon:1995}.}


\subsection{Flux passing throw a cubic block}

\subsubsection{Flux and lamination direction are parallel}
\label{sec:flux-lamin-direct}

Figure~\ref{fig:Laminated-Block-Flux-Parallel-Geometry} shows the
case where the magnetic flux is passing a laminated block in
lamination direction.  The corresponding magnetic circuit is shown in
figure~\ref{fig:Laminated-Block-Flux-Parallel-Circuit}.
\begin{figure}
  \centering
  \subfloat[Geometry]{
    \input{../images/Laminated-Block-Flux-parallel-axonometric.pdf_tex}
    \label{fig:Laminated-Block-Flux-Parallel-Geometry}
  }
  \hfil
  \subfloat[equivalent magnetic circuit]{
    \input{../images/Laminated-Block-Flux-parallel-equiv-circ.pdf_tex}
    \label{fig:Laminated-Block-Flux-Parallel-Circuit}
  }
  \caption{Magnetic flux passing a laminated block in lamination direction}
  \label{fig:Lam-Mat-Flux-parallel}
\end{figure}
\label{sec:cart-coord}
In general, the \glsdesc{Rm}~$\Rm$ of a cubic block can be expressed as
\begin{gather}
  \Rm=\frac{l\nu}{S}
\end{gather}
with
\begin{align*}
  l&\quad\text{the section's length}\text{~,}\\
  S&\quad\text{the section's cross section}\text{~,}\\
  \gls{nu}&\quad\text{the section's \glsdesc{nu}}\text{~.}
\end{align*}
\index{Magnetic resistance} For the magnetic resistors in the circuit
(figure~\ref{fig:Laminated-Block-Flux-Parallel-Circuit}) we find:
\begin{align}
  \RmAir=\frac{l\cdot\nuAir}{\SAir}\text{~,} &&
  \RmFe=\frac{l\cdot\nuFe}{\SFe}
\end{align}
with
\begin{align}
  \SFe=a\cdot \kFe \dsheet\text{~,} &&
  \SAir=a\cdot (1-\kFe) \dsheet\text{~.}
\end{align}
In these equations $\kFe$ is the stacking factor or iron fill factor.
The total magnetic resistance of these two parallel resistors is
\begin{gather}
  \label{eq:15}
\begin{split}
  \Rm&=\frac{\RmFe\cdot\RmAir}{\RmFe + \RmAir} \\
  &=\frac{l}{a\dsheet}
    \frac{1}{\frac{\kFe}{\nuFe}+\frac{1-\kFe}{\nuAir}}\text{~.}
\end{split}
\end{gather}
From equation~\eqref{eq:15} follows the homogenized reluctivty if the
flux travels the block parallel to the lamination direction as
\begin{gather}
  \nueffpar=\frac{1}{\frac{\kFe}{\nuFe}+\frac{1-\kFe}{\nuAir}}\text{~.}
\end{gather}


\subsubsection{Flux and lamination direction are perpendicular}
\label{sec:perp-lamin-direct}

Figure~\ref{fig:Lam-Mat-Flux-perpendicular} shows the geometry and
equivalent magnetic circuit for the case the magnetic flux is passing
the laminated block parallel to the lamination direction.
\begin{figure}
  \centering
  \subfloat[Geometry]{
    \input{../images/Laminated-Block-Flux-perpendicular-axonometric.pdf_tex}}
  \hfil
  \subfloat[equivalent magnetic circuit]{
    \input{../images/Laminated-Block-Flux-perpendicular-equiv-circ.pdf_tex}
  }
  \caption{Magnetic flux passing a laminated block perpendicular to lamination direction}
  \label{fig:Lam-Mat-Flux-perpendicular}
\end{figure}
The magnetic resistances in the equivalent magnetic circuit turn out to be
\begin{gather}
  \label{eq:25}
  \Rm=\frac{d\nu}{S}
\end{gather}
with
\begin{align*}
  d&\quad\text{the section's thickness}\text{~,}\\
  S&\quad\text{the section's cross section}\text{~,}\\
  \gls{nu}&\quad\text{the section's \glsdesc{nu}}\text{~.}
\end{align*}
Especially it is
\begin{align}
  \RmFe=\frac{\dsheet}{S}\cdot\kFe\nuFe\text{~,} &&
  \RmAir=\frac{\dsheet}{S}\cdot(1-\kFe)\nuAir\text{~.}
\end{align}
And the equivalent circuit's total magnetic resistance is
\begin{gather}
  \label{eq:6}
  \begin{split}
    \Rm&=\RmFe + \RmAir \\
    &=\frac{\dsheet}{S}\cdot\left[\kFe\nuFe+(1-\kFe)\nuAir\right]\text{~.}
  \end{split}
\end{gather}
From equation~\eqref{eq:6} follows the reluctivity seen by the
magnetic flux travelling the laminated block perpendicular to the
lamination direction as
\begin{gather}
  \nueffperp= \kFe\nuFe+(1-\kFe)\nuAir \text{~.}
\end{gather}


\subsection{Flux passing through a toroidal Block}
\label{sec:cylinder-coordinates}

\begin{figure}
  \centering
  \input{../images/Laminated-Block-Toroidal.pdf_tex}
  \caption{Toroidal laminated block}
  \label{fig:Toroidal-laminated-block}
\end{figure}
First, we want to derive the magnetic resistance passing the block in
figure~\ref{fig:Toroidal-laminated-block} in radial direction:
\begin{gather}
  \Rm=\frac{\Um}{\Phi}\text{~,} \\
  \Um=Hr(\alpha_2-\alpha_1)\text{~,} \\
  \Phi=\iinteg{S}{\vect{B}}{\vect{S}}
  =\frac{h\Um}{\alpha_2-\alpha_1}
  \integg{\rin}{\rout}{\frac{\mu(r)}{r}}{r}\text{~.}
  \intertext{It follows:}
  \label{eq:78}
  \Rm=\frac{\alpha_2-\alpha_1}{h}\left(
    \integg{\rin}{\rout}{\frac{\mu(r)}{r}}{r}\right)^{-1}\text{~.}
\end{gather}
When passing this block in angular direction, the magnetic flux
experiences the following resistance:
\begin{gather}
  \Rm=\frac{\Um}{\Phi}\text{~,} \\
  B=\frac{\Phi}{hr(\alpha_2-\alpha_1)}\text{~,} \\
  \Um =\integ{}{\vect{H}}{\vect{l}}
  =\frac{\Phi}{h(\alpha_2-\alpha_1)}
  \integg{\rin}{\rout}{\frac{1}{r\cdot\mu(r)}}{r}\text{~.}
  \intertext{It follows:}
  \label{eq:77}
  \Rm=\frac{1}{h(\alpha_2-\alpha_1)}
  \integg{\rin}{\rout}{\frac{1}{r\cdot\mu(r)}}{r}\text{~.}
\end{gather}
Equation \eqref{eq:78} and \eqref{eq:77} give the exact solutions for
the case the magnetic flux is passing a toroidal block in or
perpendicular to the lamination direction.  This solution is not a
very practical solution:  Because the material's permeability depends
on the magnetic flux density which is inhomogeniously distributed, the
material must also be seen as an inhomogenious material.  Therefore it
is impossible to take the material's permeability $\mu(r)$ out
of the integral.  But what are we really interested in?  We're
interested in cases where \rin and \rout differ only
by the sheet's thickness; that means cases where
$\rin\approx\rout$.  In these cases we can assume,
that the magnetic flux density~$B$ and therefore also the
permeability~$\mu$ is almost homogeneous in the computation domain.
With this precondition it follows, that the permeability is the same
as in the Cartesian coordinate case.  When being in doubt that this
precondition is valid it is of course possible to select a higher
number of vertices in radial direction.


\section{Nonlinear timeharmonic computations}
\label{sec:nonl-timeh-comp}
\index{Computation!nonlinear timeharmonic|emph}

A timeharmonic computation implies that all quantities have a
sinusoidal waveform.  On the other hand material non-linearities
introduce a notable distortion.  This is shown in
figure~\ref{fig:waveform-distortion} for a typical electrical steel
\footnote{The material used as an example throughout this section is a
non oriented and fully processed electrical steel: M330-35A.  The
material data is taken from \cite{Cogent:NOFP}.}.
\begin{figure}
  \begin{maxipage}
  \centering
  \subfloat[excitation by a sinusoidal current]{
    \begin{tikzpicture}
      \begin{axis}[
        height=.44\textheight,
        tick style={line width=.18mm},
        view={132}{30},
        domain=0:pi,
        xlabel=$\frac{\omega t}{\pi}$,
        ylabel=$H\cdot\frac{\metre}{\kilo\ampere}$,
        zlabel=${B}/{\tesla}$,
        line width=.35mm]
        \addplot3[mark=none,line width=.7mm,style=dotted] file {../datasets/M330-35-A.csv};
        \addplot3[mark=none,line width=.7mm,style=loosely dashdotted]
        ({y/pi}, {.16*sin(deg(y))}, {0});
        \addplot3[mark=none,line width=.7mm,style=densely dashdotted]
        ({y/pi}, {2.8*sin(deg(y))}, {0});
        \addplot3[mark=none,line width=.7mm,style=loosely dashdotted]
        file {../datasets/M330-35-A_H-Ampl-160.csv};
        \addplot3[mark=none,line width=.7mm,style=densely dashdotted]
        file {../datasets/M330-35-A_H-Ampl-2800.csv};
      \end{axis}
    \end{tikzpicture}
  }\\[\fill]
  \subfloat[excitation by a sinusoidal voltage]{
    \begin{tikzpicture}
      \begin{axis}[
        height=.44\textheight,
        tick style={line width=.35mm},
        view={132}{30},
        domain=0:pi,
        xlabel=$\frac{\omega t}{\pi}$,
        ylabel=$H\cdot\frac{\metre}{\kilo\ampere}$,
        zlabel=${B}/{\tesla}$,
        line width=.35mm]
        \addplot3[mark=none,line width=.7mm,style=dotted] 
        file {../datasets/M330-35-A.csv};
        \addplot3[mark=none,line width=.7mm,style=loosely dashdotted]
        file {../datasets/M330-35-A_B-Ampl-0.6.csv};
        \addplot3[mark=none,line width=.7mm,style=densely dashdotted]
        file {../datasets/M330-35-A_B-Ampl-1.6.csv};
        \addplot3[mark=none,line width=.7mm,style=loosely dashdotted]
        ({y/pi}, {0}, {.6*sin(deg(y))});
        \addplot3[mark=none,line width=.7mm,style=densely dashdotted]
        ({y/pi}, {0}, {1.6*sin(deg(y))});
      \end{axis}
    \end{tikzpicture}
  }
  \caption{Waveform distortion due to nonlinear material properties}
  \label{fig:waveform-distortion}
  \end{maxipage}
\end{figure}
In these figures the $H$-$B$-plane shows the original
magnetization curve and the other planes show one of the quantities
$B$ and $H$ over time when they are mirrored at the
magnetization curve.  \par So, how can despite these non-linearities a
timeharmonic computation be done?  In the presence of non-linearities
the goal of a timeharmonic computation isn't anymore to match the
exact waveform -- in the presence of non-linearities the goal is to
match the distorted waveform's fundamental.  To reach this goal the
magnetization curve is transformed.  First we observe that in the
case of sinusoidal quantities:
\begin{align}
  \label{eq:63}
  B&\sim U &&
  \text{because}&&\Phi=\integ{}{U}{t}
  \intertext{and} %%
  H&\sim I && \text{because}&&H\cdot l=\Theta
\end{align}
From that it follows that depending on whether the system is excited
by a sinusoidal current or a sinusoidal voltage the magnetization
curve must be transformed in a different way.  In the case of
excitation by a sinusoidal current the transformation is
carried out in the following way:
\begin{enumerate}
\item We assume that we start with values measured by the material
  supplier.  Typically the peak field intensity and the peak
  polarization are provided.
\item As stated in equation~\eqref{eq:63} the flux density and not the
  polarization is required:
  \begin{gather}
    \label{eq:64}
    B = B_{\mathrm{i}}+\mu_0 H
  \end{gather}
\item The resulting $(H;B)$-value-pairs are interpolated by a
  cubic spline (or some other suitable interpolating function).
\item From the provided field intensity interval a set of discrete
  values is chosen.  For each of these values $\hat{H}$ we compute:
  \begin{enumerate}
  \item $H(\omega t)=\hat{H}cos(\omega t)$ with $\omega
    t\in[0;\pi[$
  \item $B(\omega t)=B(H(\omega t))$
  \item Find the fundamental~$\hat{B}$ of $B(\omega t)$ by
    using the Fourier transformation
  \item $(\hat{H};\hat{B})$ is one value pair of the
    transformed curve
  \end{enumerate}
\end{enumerate}
In the case of excitation by a sinusoidal voltage $H$ and $B$
must be interchanged.  Figure~\ref{fig:Transformed-mag-curves} shows
the result of this procedure for the material M330-35-A.
\begin{figure}
  \begin{maxipage}
  \centering
  \subfloat[transformed curve for current input]{
    \begin{tikzpicture}
      \begin{semilogxaxis}[
%        height=.44\textheight,
%        width=.7119349550499537\textheight,
        width=.8\textwidth,
        height=.49442719099991583\textwidth,
        xlabel=$H\cdot\frac{\metre}{\kilo\ampere}\quad\longrightarrow$,
        ylabel=${B}/{\tesla}\quad\longrightarrow$,
        grid=major,
        grid style={line width=.18mm},
        tick style={line width=.35mm},
        legend style={anchor=south east,at={(.92,.08)}},
        line width=.35mm]
        \addplot[mark=none,line width=.7mm,style=dotted]
        file {../datasets/M330-35-A_xy.csv};
        \addplot[mark=none,line width=.7mm,style=loosely dashed]
        file {../datasets/M330-35-A_current-input.csv};
        \legend{orig. curve, transformed}
      \end{semilogxaxis}
    \end{tikzpicture}
  } \\[\fill]
  \subfloat[transformed curve for voltage input]{
    \begin{tikzpicture}
      \begin{semilogxaxis}[
%        height=.44\textheight,
%        width=.7119349550499537\textheight,
        width=.8\textwidth,
        height=.49442719099991583\textwidth,
        xlabel=$H\cdot\frac{\metre}{\kilo\ampere}\quad\longrightarrow$,
        ylabel=${B}/{\tesla}\quad\longrightarrow$,
        grid=major,
        grid style={line width=.18mm},
        tick style={line width=.35mm},
        legend style={anchor=south east,at={(.92,.08)}},
        line width=.35mm]
        \addplot[mark=none,line width=.7mm,style=dotted]
        file {../datasets/M330-35-A_xy.csv};
        \addplot[mark=none,line width=.7mm,style=loosely dashed]
        file {../datasets/M330-35-A_voltage-input.csv};
        \legend{orig. curve, transformed}
      \end{semilogxaxis}
    \end{tikzpicture}
}
  \caption{Transformed magnetization curves}
  \label{fig:Transformed-mag-curves}
  \end{maxipage}
\end{figure}

%%% Local Variables: 
%%% mode: latex
%%% TeX-master: "manual"
%%% ispell-local-dictionary: "en_US"
%%% End: 
