%%% -*- mode: LaTeX -*-
%
% Copyright © 2012-2016 Philipp Büttgenbach
%
% ulphi.en.tex is part of ulphi, a CAE tool and library for computing
% electromagnetic fields.
%
% All rights reserved.
%
% This program is distributed in the hope that it will be useful, but
% WITHOUT ANY WARRANTY; without even the implied warranty of
% MERCHANTABILITY or FITNESS FOR A PARTICULAR PURPOSE.
%


\title{Compute Magnetic Fields with Ulphi}
\author{Philipp Büttgenbach}
\institute{Philipp.Buettgenbach@gmx.de}
\subject{Compute Magnetic Fields}


% If you have a file called "university-logo-filename.xxx", where xxx
% is a graphic format that can be processed by latex or pdflatex,
% resp., then you can add a logo as follows:

%\pgfdeclareimage[height=0.5cm]{ulphi-logo}{../images/Ulphi-logo}
%\logo{\pgfuseimage{ulphi-logo}}


% If you wish to uncover everything in a step-wise fashion, uncomment
% the following command: 

%\beamerdefaultoverlayspecification{<+->}


\begin{document}

%%% -*- mode: LaTeX -*-
%
% Copyright © 2012-2016 Philipp Büttgenbach
%
% ulphi.en.tex is part of ulphi, a CAE tool and library for computing
% electromagnetic fields.
%
% All rights reserved.
%
% This program is distributed in the hope that it will be useful, but
% WITHOUT ANY WARRANTY; without even the implied warranty of
% MERCHANTABILITY or FITNESS FOR A PARTICULAR PURPOSE.
%

\begin{frame}
  \titlepage
\end{frame}

%%% Local Variables: 
%%% mode: latex
%%% TeX-master: nil
%%% ispell-local-dictionary: "en_US"
%%% End: 

\include{frame02}

\section{About Ulphi}

%%% -*- mode: LaTeX -*-
%
% Copyright © 2012-2016 Philipp Büttgenbach
%
% ulphi.en.tex is part of ulphi, a CAE tool and library for computing
% electromagnetic fields.
%
% All rights reserved.
%
% This program is distributed in the hope that it will be useful, but
% WITHOUT ANY WARRANTY; without even the implied warranty of
% MERCHANTABILITY or FITNESS FOR A PARTICULAR PURPOSE.
%

\begin{frame}{What is it?}
  \begin{itemize}
  \item Compute 2D magnetic fields
  \item Static and quasistatic fields supported
  \item Based on finite integration technique (FIT)
  \item Targets at models with a rather simple geometry
  \item Couple polar and cartesian sections in one model
  \item Strong support for nonlinear and anisotropic materials
  \end{itemize}
  \vfill
  \centering
  \includeinkscape[height=.25\textheight]{Ulphi-logo}
\end{frame}

%%% Local Variables: 
%%% mode: latex
%%% TeX-master: nil
%%% ispell-local-dictionary: "en_US"
%%% End: 

%%% -*- mode: LaTeX -*-
%
% Copyright © 2012-2016 Philipp Büttgenbach
%
% ulphi.en.tex is part of ulphi, a CAE tool and library for computing
% electromagnetic fields.
%
% All rights reserved.
%
% This program is distributed in the hope that it will be useful, but
% WITHOUT ANY WARRANTY; without even the implied warranty of
% MERCHANTABILITY or FITNESS FOR A PARTICULAR PURPOSE.
%

\begin{frame}{Features}
  \begin{itemize}
  \item Coded in modern C++
  \item Python interface for everyday use
  \item Use the power of scipy, matplotlib and other packages for
    postprocessing
  \item Units must be stated explicitly
  \item High performance equation solving by \texttt{libeigen}
%  \item Nonlinear equation system solving with Newton's procedure
  \end{itemize}
\end{frame}

%%% Local Variables: 
%%% mode: latex
%%% TeX-master: nil
%%% ispell-local-dictionary: "en_US"
%%% End: 


\section{Ulphi in Detail}
\subsection{The Finite Integration Technique}

%%% -*- mode: LaTeX -*-
%
% Copyright © 2012-2016 Philipp Büttgenbach
%
% ulphi.en.tex is part of ulphi, a CAE tool and library for computing
% electromagnetic fields.
%
% All rights reserved.
%
% This program is distributed in the hope that it will be useful, but
% WITHOUT ANY WARRANTY; without even the implied warranty of
% MERCHANTABILITY or FITNESS FOR A PARTICULAR PURPOSE.
%

\begin{frame}{Equations in Magnetostatics}
  \begin{align*}
    \label{eq:1}
    \nabla\times\left(\nu\vect{B}\right) &= \vect{J} &\Leftrightarrow&&
    \ointeg{l}{\nu\vect{B}}{\vect{l}}&= \Theta \\
    %%
    \nabla\times\vect{A} &= \vect{B} &\Leftrightarrow&&
    \ointeg{l}{\vect{B}}{\vect{l}}&= \Phi
  \end{align*}
  \begin{itemize}
  \item Two equations require two grids
  \item Grids are made up of the integral's integration pathes
  \end{itemize}
\end{frame}

%%% Local Variables: 
%%% mode: latex
%%% TeX-master: nil
%%% ispell-local-dictionary: "en_US"
%%% End: 

%%% -*- mode: LaTeX -*-
%
% Copyright © 2012-2016 Philipp Büttgenbach
%
% ulphi.en.tex is part of ulphi, a CAE tool and library for computing
% electromagnetic fields.
%
% All rights reserved.
%
% This program is distributed in the hope that it will be useful, but
% WITHOUT ANY WARRANTY; without even the implied warranty of
% MERCHANTABILITY or FITNESS FOR A PARTICULAR PURPOSE.
%

\begin{frame}{A basic FIT Grid}
  \centering
  \includeinkscape{FIT-grid-3D}
\end{frame}

%%% Local Variables: 
%%% mode: latex
%%% TeX-master: nil
%%% ispell-local-dictionary: "en_US"
%%% End: 


\subsection{Boundary Conditions}

%%% -*- mode: LaTeX -*-
%
% Copyright © 2012-2016 Philipp Büttgenbach
%
% ulphi.en.tex is part of ulphi, a CAE tool and library for computing
% electromagnetic fields.
%
% All rights reserved.
%
% This program is distributed in the hope that it will be useful, but
% WITHOUT ANY WARRANTY; without even the implied warranty of
% MERCHANTABILITY or FITNESS FOR A PARTICULAR PURPOSE.
%

\begin{frame}{Boundary condition}{Neumann}
  \begin{itemize}
  \item Angle between boundary and flux direction $\alpha<\pi$
  \item Homogenious Neumann condition arises naturally
  \end{itemize}
  \vfill
  \centering
  \input{../images/Neumann-Boundary.pdf_tex}
\end{frame}

%%% Local Variables: 
%%% mode: latex
%%% TeX-master: nil
%%% ispell-local-dictionary: "en_US"
%%% End: 

%%% -*- mode: LaTeX -*-
%
% Copyright © 2012-2016 Philipp Büttgenbach
%
% ulphi.en.tex is part of ulphi, a CAE tool and library for computing
% electromagnetic fields.
%
% All rights reserved.
%
% This program is distributed in the hope that it will be useful, but
% WITHOUT ANY WARRANTY; without even the implied warranty of
% MERCHANTABILITY or FITNESS FOR A PARTICULAR PURPOSE.
%

\begin{frame}{Boundary condition}{Dirichlet}
  \begin{itemize}
  \item Flux direction is parallel to boundary
  \item May be used to specify field sources
  \item A time varying vector potential is supported
  \end{itemize}
  \vfill
  \centering
  \input{../images/Dirichlet-Boundary.pdf_tex}
\end{frame}

%%% Local Variables: 
%%% mode: latex
%%% TeX-master: nil
%%% ispell-local-dictionary: "en_US"
%%% End: 


\subsection{Grid Coupling}

\include{frame09}

\section{Examples}

\subsection{Coaxial line}

%%% -*- mode: LaTeX -*-
%
% Copyright © 2012-2016 Philipp Büttgenbach
%
% ulphi.en.tex is part of ulphi, a CAE tool and library for computing
% electromagnetic fields.
%
% All rights reserved.
%
% This program is distributed in the hope that it will be useful, but
% WITHOUT ANY WARRANTY; without even the implied warranty of
% MERCHANTABILITY or FITNESS FOR A PARTICULAR PURPOSE.
%

\begin{frame}{Coaxial line}{Analytical solution}
  \begin{multline*}
    A(0)-A(r) = \frac{\mu_0J_1}{2} \\ \cdot
    \begin{cases}
      r^2/2 & r \le r_1\text{~,} \\
      r_1^2\left(\frac{1}{2}+\ln\frac{r}{r_1}\right)
      & r_1 < r \le r_2\text{~,} \\
      r_1^2\left(\frac{1}{2}+\ln\frac{r_2}{r_1}
        +\left(1+\frac{r_2^2}{r_3^2-r_2^2}\right)\ln\frac{r}{r_2}
        +\frac{r^2-r_2^2}{2(r_3^2-r_2^2)}\right)
      & r_2 < r \le r_3\text{~.} \\
    \end{cases}
  \end{multline*}
  \vfill
  \centering
  \input{../images/Coaxial-Line.pdf_tex}
\end{frame}

%%% Local Variables: 
%%% mode: latex
%%% TeX-master: nil
%%% ispell-local-dictionary: "en_US"
%%% End: 

%%% -*- mode: LaTeX -*-
%
% Copyright © 2012-2016 Philipp Büttgenbach
%
% ulphi.en.tex is part of ulphi, a CAE tool and library for computing
% electromagnetic fields.
%
% All rights reserved.
%
% This program is distributed in the hope that it will be useful, but
% WITHOUT ANY WARRANTY; without even the implied warranty of
% MERCHANTABILITY or FITNESS FOR A PARTICULAR PURPOSE.
%

\begin{frame}{Coaxial line}{Results}
  \begin{itemize}
  \item Numerical and analytical solution match very well.
  \end{itemize}
  \vfill
  \centering
  \begin{tikzpicture}
    \begin{axis}[
      width=.9\linewidth,
      height=.5562305898749054\linewidth,
      xlabel=${r}/\unit{\milli\metre}\quad\longrightarrow$,
      ylabel=$A\cdot\unit{\metre\per\micro\weber}\quad\longrightarrow$,
      grid=major,
      grid style={line width=.18mm},
      tick style={line width=.35mm},
      legend style={at={(0.95,0.95)},anchor=north east},
      line width=.35mm]
      \addplot[mark=diamond,only marks]
        table[x=x, y=NumA] {coaxial_line.csv};
      \addplot[smooth,mark=none,line width=.7mm,style=dotted]
        table[x=x, y=SymA] {coaxial_line.csv};
      \legend{numerical, analytical}
    \end{axis}
  \end{tikzpicture}
  \vfil
\end{frame}

%%% Local Variables: 
%%% mode: latex
%%% TeX-master: nil
%%% ispell-local-dictionary: "en_US"
%%% End: 


\subsection{Simple iron core}

%%% -*- mode: LaTeX -*-
%
% Copyright © 2012-2016 Philipp Büttgenbach
%
% ulphi.en.tex is part of ulphi, a CAE tool and library for computing
% electromagnetic fields.
%
% All rights reserved.
%
% This program is distributed in the hope that it will be useful, but
% WITHOUT ANY WARRANTY; without even the implied warranty of
% MERCHANTABILITY or FITNESS FOR A PARTICULAR PURPOSE.
%

\begin{frame}{Simple iron core}
  Check of average flux density:
  \begin{center}
    \emph{Ulphi:} \qty{1.66}{\tesla} \hfil
    \emph{Average iron path:} \qty{1.59}{\tesla}
  \end{center}
  \begin{itemize}
  \item Ulphi can be trusted
  \end{itemize}
  \vfil
  \centering
  \begin{tikzpicture}
    \begin{axis}[
      title={$A\cdot\unit{\metre\per\milli\weber}$},
      height=.6\textheight,
      axis equal={true},
      xmin={0},
      xmax=50,
      ymin=-5,
      ymax=60,
      xtick={-10,0,...,60},
      ytick={-10,0,...,60},
      xlabel=$x/\unit{\milli\metre}\quad\longrightarrow$,
      ylabel=$y/\unit{\milli\metre}\quad\longrightarrow$,
      grid=major,
      grid style={line width=.18mm},
      tick style={line width=.35mm},
      line width=.35mm]
      %% iron core shape
      \addplot[gray,fill=gray!20,opacity=.33,line width=.7mm] 
        coordinates {
          (15,0) (40,0) (40,54) (0, 54) (0, 29) (15,29)} -- cycle;
      %% coils
      \addplot[copper,fill=copper,opacity=.33,line width=.7mm]
        coordinates {(5,0) (10,0) (10,24) (5,24)} -- cycle;
      \addplot[copper,fill=copper,opacity=.33,line width=.7mm]
        coordinates { (45,0) (50,0) (50,24) (45,24)} -- cycle;
      % \addplot[contour gnuplot, line width=.7mm]
      %   table[x=x, y=y, z=potential] {simple_iron_core.csv};
    \end{axis}
  \end{tikzpicture}
\end{frame}

%%% Local Variables: 
%%% mode: latex
%%% TeX-master: nil
%%% ispell-local-dictionary: "en_US"
%%% End: 


\subsection{Wound iron core}

%%% -*- mode: LaTeX -*-
%
% Copyright © 2012-2016 Philipp Büttgenbach
%
% ulphi.en.tex is part of ulphi, a CAE tool and library for computing
% electromagnetic fields.
%
% All rights reserved.
%
% This program is distributed in the hope that it will be useful, but
% WITHOUT ANY WARRANTY; without even the implied warranty of
% MERCHANTABILITY or FITNESS FOR A PARTICULAR PURPOSE.
%

\begin{frame}{Wound iron core}{Model}
  \begin{itemize}
  \item Make use of the grid coupling facility
  \item Make use of the anisotropic material models
  \end{itemize}
  \vfil
  \begin{minipage}{\linewidth}
    \hfil
    \input{../images/Wound-Transformer-Core.pdf_tex}
    \hfil
    \input{../images/wound-iron-core-model.pdf_tex}
    \hfil
  \end{minipage}
\end{frame}

%%% Local Variables: 
%%% mode: latex
%%% TeX-master: nil
%%% ispell-local-dictionary: "en_US"
%%% End: 

%%% -*- mode: LaTeX -*-
%
% Copyright © 2012-2016 Philipp Büttgenbach
%
% ulphi.en.tex is part of ulphi, a CAE tool and library for computing
% electromagnetic fields.
%
% All rights reserved.
%
% This program is distributed in the hope that it will be useful, but
% WITHOUT ANY WARRANTY; without even the implied warranty of
% MERCHANTABILITY or FITNESS FOR A PARTICULAR PURPOSE.
%

\begin{frame}{Wound iron core}{Result}
  \raggedright
  \begin{tikzpicture}
    \begin{axis}[ width=.7119349550499537\textheight,
      height=.44\textheight,
      xlabel=$t/{\milli\second}\quad\longrightarrow$,
      ylabel=$\Um/{\kilo\ampere}\quad\longrightarrow$, grid=major, grid
      style={line width=.18mm}, tick style={line width=.35mm},
      legend style={at={(0.95,0.95)},anchor=north east}, line
      width=.35mm]
      \addplot[line width=.7mm] table[x=t,
      y=y1]{../datasets/wound_iron_core_time.csv}; 
      \addplot[smooth, line
      width=.7mm, style=dashed] table[x=t,
      y=y2]{../datasets/wound_iron_core_time.csv}; \legend{waveform,
        fundamental}
    \end{axis}
  \end{tikzpicture}
  \vfill
  \raggedleft
  \begin{tikzpicture}
    \begin{axis}[ width=.7119349550499537\textheight,
      height=.44\textheight, xlabel=harmonic
      order$\quad\longrightarrow$,
      ylabel=$\Um/{\kilo\ampere}\quad\longrightarrow$, grid=major, grid
      style={line width=.18mm}, tick style={line width=.35mm},
      line width=.35mm]
      \addplot[ycomb, line width=.7mm]
      table{../datasets/wound_iron_core_freq.csv};
    \end{axis}
  \end{tikzpicture}
\end{frame}

%%% Local Variables: 
%%% mode: latex
%%% TeX-master: nil
%%% ispell-local-dictionary: "en_US"
%%% End: 


\section{Conclusion}

%%% -*- mode: LaTeX -*-
%
% Copyright © 2012-2016 Philipp Büttgenbach
%
% ulphi.en.tex is part of ulphi, a CAE tool and library for computing
% electromagnetic fields.
%
% All rights reserved.
%
% This program is distributed in the hope that it will be useful, but
% WITHOUT ANY WARRANTY; without even the implied warranty of
% MERCHANTABILITY or FITNESS FOR A PARTICULAR PURPOSE.
%

\begin{frame}{Conclusion}
  \begin{itemize}
  \item Compute 2D static and quasistatic magnetic fields
  \item Developed with the modelling demands of wound iron cores in
    mind
    \begin{itemize}
    \item Couple cartesian and polar grids
    \item Anisotropic materials
    \item Avoid interpolation of material properties
    \end{itemize}
  \item Python interface for everyday use
  \item Use scipy and matplotlib for postprocessing
  \end{itemize}
\end{frame}

%%% Local Variables: 
%%% mode: latex
%%% TeX-master: nil
%%% ispell-local-dictionary: "en_US"
%%% End: 


\end{document}

%%% Local Variables: 
%%% mode: latex
%%% TeX-master: nil
%%% ispell-local-dictionary: "en_US"
%%% End: 
